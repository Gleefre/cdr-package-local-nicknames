% Created 2024-07-05 Fri 13:32
% Intended LaTeX compiler: pdflatex
\documentclass[11pt]{article}
\usepackage[utf8]{inputenc}
\usepackage[T1]{fontenc}
\usepackage{graphicx}
\usepackage{longtable}
\usepackage{wrapfig}
\usepackage{rotating}
\usepackage[normalem]{ulem}
\usepackage{amsmath}
\usepackage{amssymb}
\usepackage{capt-of}
\usepackage{hyperref}
\usepackage[margin=1in]{geometry}
\author{Gleefre}
\date{\today}
\title{PLN CDR draft: Issue 8}
\hypersetup{
 pdfauthor={Gleefre},
 pdftitle={PLN CDR draft: Issue 8},
 pdfkeywords={},
 pdfsubject={},
 pdfcreator={Emacs 28.2 (Org mode 9.5.5)}, 
 pdflang={English}}
\begin{document}

\maketitle

\section{Issue 8 (Interaction with FORMAT)}
\label{sec:orgfb20ba9}
\emph{by |3b|}
\subsection{Description}
\label{sec:orgffdabcc}
It is not clear how \emph{local nicknames} should affect \texttt{format}'s \texttt{\textbackslash{}\textasciitilde{}//} directive.

First, it would be inconvinient if it wouldn't be possible to use \emph{local
nicknames} with the \texttt{\textbackslash{}\textasciitilde{}//} directive.

Secondly, it would be unintuitive if the function used would depend on the
\emph{current package} at the \textbf{execution} time. This also might break [existing] code,
if a local nickname in the \emph{current package} shadows a package that contains a
function used by a control string in another function.

Finally, if the call to \texttt{format} is not compiled, it is hard to impossible to find
the function using \emph{local nicknames} of the package that was the \emph{current package}
at the \textbf{compile} time.
\subsection{Examples}
\label{sec:org44250f6}
\begin{verbatim}
(defpackage #:foo-a (:use) (:export #:ff))
(defpackage #:foo-b (:use) (:export #:ff))

(defun foo-a:ff (stream &rest args)
  (declare (ignore args))
  (format stream "FOO-A:FF"))

(defun foo-b:ff (stream &rest args)
  (declare (ignore args))
  (format stream "FOO-B:FF"))

(defpackage #:bar-a
  (:use #:cl)
  (:local-nicknames (#:nick #:foo-a)))

(defpackage #:bar-b
  (:use #:cl)
  (:local-nicknames (#:nick #:foo-b)))

(in-package #:bar-a)

(defun test ()
  (format t "Called ~/nick:ff/ & " nil)
  (let ((*package* (find-package (quote #:bar-a))))  ; or #.*package*
    (format t "~/nick:ff/~%" nil)))

(test)
; => "Called FOO-A:FF & FOO-A:FF"  (sbcl, ccl, ecl, acl, abcl, clasp)
; lispworks errors (NICK package not found)
(let ((*package* (find-package (quote #:bar-b))))
  (test))
; => "Called FOO-A:FF & FOO-A:FF"  (sbcl, clasp)
; => "Called FOO-B:FF & FOO-A:FF"  (ccl, ecl, acl, abcl)
; lispworks errors (NICK package not found)
\end{verbatim}
\subsection{Proposal NO-LOCAL-NICKNAMES}
\label{sec:org24e7cde}
\texttt{format}'s \texttt{\textbackslash{}\textasciitilde{}//} directive must \textbf{not} use \emph{local nicknames} of any package when
looking up the specified symbol.

Rationale: In the spirit of how when the package is not specified, the symbol is
not looked up in the \emph{current package}, but instead in the \texttt{\#:CL-USER} package;
the \emph{tilde slash} directive should not depend on the value of \texttt{*package*} at any
time.  Specifying it to use \emph{local nicknames} of the \texttt{\#:CL-USER} package instead
would risk breaking the existing code when adding local nicknames to that package.
\subsection{Links}
\label{sec:org1a644c2}
See \href{https://www.lispworks.com/documentation/HyperSpec/Body/22\_ced.htm}{CLHS 22.3.5.4 Tilde Slash: Call Function}.
\end{document}