% Created 2024-07-06 Sat 17:24
% Intended LaTeX compiler: pdflatex
\documentclass[11pt]{article}
\usepackage[utf8]{inputenc}
\usepackage[T1]{fontenc}
\usepackage{graphicx}
\usepackage{longtable}
\usepackage{wrapfig}
\usepackage{rotating}
\usepackage[normalem]{ulem}
\usepackage{amsmath}
\usepackage{amssymb}
\usepackage{capt-of}
\usepackage{hyperref}
\usepackage[margin=1in]{geometry}
\author{Gleefre}
\date{\today}
\title{PLN CDR draft: Issue 1}
\hypersetup{
 pdfauthor={Gleefre},
 pdftitle={PLN CDR draft: Issue 1},
 pdfkeywords={},
 pdfsubject={},
 pdfcreator={Emacs 28.2 (Org mode 9.5.5)}, 
 pdflang={English}}
\begin{document}

\maketitle

\section{Issue 1 (ADD-/REMOVE-PACKAGE-LOCAL-NICKNAME return values)}
\label{sec:org15469b4}
\subsection{Description}
\label{sec:orgcf78821}
Functions \texttt{add-package-local-nickname} and \texttt{remove-package-local-nickname} have
inconsistent return values.

The first one always returns the \emph{designated package}, while the second one
returns \emph{true} (a \emph{generalized boolean}) that indicates whether a nickname was
removed.

Furthermore, there is no consensus among existing implementations as to what this
\emph{generalized boolean} represents.

In comparison,the functions \texttt{use-package} and \texttt{unuse-package} always return \texttt{T},
and the functions \texttt{export} and \texttt{unexport} have the same behavior. There are also
functions \texttt{unintern} and \texttt{delete-package} which return a \emph{generalized boolean}
indicating if the operation changed something, but their counterparts \texttt{intern} and
\texttt{make-package}, unlike with local nicknames, create and return an \emph{object} (a
\emph{symbol} or a \emph{package}).
\subsection{Examples}
\label{sec:org77daeb1}
\begin{verbatim}
(defpackage #:foo (:use))
(defpackage #:bar (:use))

(add-package-local-nickname '#:nick '#:bar '#:foo)
; => #<PACKAGE "FOO">  (sbcl, ccl, ecl, acl, abcl, clasp, lispworks)
(add-package-local-nickname '#:nick '#:bar '#:foo)
; => #<PACKAGE "FOO">  (sbcl, ccl, ecl, acl, abcl, clasp, lispworks)
(remove-package-local-nickname '#:nick '#:foo)
; => T  (sbcl, ccl, ecl, acl, clasp)
; => #<PACKAGE "BAR">  (abcl)
; => NIL  (lispworks)
(remove-package-local-nickname '#:nick '#:foo)
; => NIL  (sbcl, ccl, ecl, acl, abcl, clasp, lispworks)
\end{verbatim}
\subsection{Current behavior}
\label{sec:org23ae73d}
sbcl, ccl, ecl, acl, abcl, clasp, lispworks:
  \texttt{add-package-local-nickname} always returns \emph{designated package}.

sbcl, ccl, ecl, acl, clasp:
  \texttt{remove-package-local-nickname} returns \texttt{T} if a nickname was removed,
  and \texttt{NIL} otherwise.

abcl:
  \texttt{remove-package-local-nickname} returns \emph{true} if a nickname was removed, (more
  specifically, it returns the package nicknamed by the removed local nickname),
  and \texttt{NIL} otherwise.

lispworks:
  \texttt{remove-package-local-nickname} always returns \texttt{NIL}.
\subsection{Proposal ALWAYS-T}
\label{sec:org34cbb92}
\texttt{add-package-local-nickname} should always return \texttt{T}.

\texttt{remove-package-local-nickname} should always return \texttt{T}.
\subsection{Proposal ELIMINATE-GENERALIZED-BOOLEAN}
\label{sec:orgd5a5c1f}
\texttt{add-package-local-nickname} should return the \emph{designated package}.

\texttt{remove-package-local-nickname} should return \texttt{T} if a nickname was removed and
\texttt{NIL} otherwise.
\subsection{Proposal DESIGNATED-PACKAGE-IF-SUCCESSFUL}
\label{sec:org6f252c4}
\texttt{add-package-local-nickname} should return the \emph{designated package} if a new
nickname was added, and \texttt{NIL} otherwise.

\texttt{remove-package-local-nickname} should return the \emph{designated package} if a
nickname was removed, and \texttt{NIL} otherwise.
\subsection{Proposal NICKNAMED-PACKAGE-IF-SUCCESSFUL}
\label{sec:org979c9c1}
\texttt{add-package-local-nickname} should return the \emph{nicknamed package} if a new
nickname was added, and \texttt{NIL} otherwise.

\texttt{remove-package-local-nickname} should return the previously \emph{nicknamed package}
if a nickname was removed, and \texttt{NIL} otherwise.
\end{document}