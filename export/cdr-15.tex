% Created 2024-07-05 Fri 13:43
% Intended LaTeX compiler: pdflatex
\documentclass[11pt]{article}
\usepackage[utf8]{inputenc}
\usepackage[T1]{fontenc}
\usepackage{graphicx}
\usepackage{longtable}
\usepackage{wrapfig}
\usepackage{rotating}
\usepackage[normalem]{ulem}
\usepackage{amsmath}
\usepackage{amssymb}
\usepackage{capt-of}
\usepackage{hyperref}
\usepackage[margin=1in]{geometry}
\usepackage[margin=1in]{geometry}
\usepackage[margin=1in]{geometry}
\usepackage[margin=1in]{geometry}
\usepackage[margin=1in]{geometry}
\usepackage[margin=1in]{geometry}
\usepackage[margin=1in]{geometry}
\usepackage[margin=1in]{geometry}
\usepackage[margin=1in]{geometry}
\usepackage[margin=1in]{geometry}
\usepackage[margin=1in]{geometry}
\author{Gleefre}
\date{\today}
\title{Package-Local Nicknames}
\hypersetup{
 pdfauthor={Gleefre},
 pdftitle={Package-Local Nicknames},
 pdfkeywords={},
 pdfsubject={This is a CDR specification for package-local nicknames.},
 pdfcreator={Emacs 28.2 (Org mode 9.5.5)}, 
 pdflang={English}}
\begin{document}

\maketitle
\tableofcontents

[THIS IS A DRAFT]

\section{Specification}
\label{sec:org4acebec}
\href{./spec.org}{Specification}

\subsection{Introduction}
\label{sec:org7514393}
This is a specification for the package-local nicknames extension in Common Lisp.
\subsubsection{Rationale}
\label{sec:orgf1bcc04}
Package-local nicknames allow to use short and easy-to-use names for packages
locally without potentially introducing name conflict as with normal nicknames.
\subsubsection{Current state}
\label{sec:org9b7facb}
Package-local nicknames are implemented in some form in \texttt{SBCL}, \texttt{CCL}, \texttt{ECL},
\texttt{Clasp}, \texttt{ABCL}, \texttt{Allegro CL}, \texttt{LispWorks}. There also exists a pending PR for the
\texttt{CLISP} implementation.

Unfortunately, there are multiple inconsistencies between implementations, all
implementations lose the \textbf{print-read} consistency to some extent, and there are
multiple edge cases that aren't always implemented correctly.
\subsubsection{Goal}
\label{sec:org9b12017}
The purpose of this document is to standardize the package-local nicknames
extension and to address some existing issues.

[TODO] This CDR also aims to provide an extensive test suite for this extension.
\subsection{Description}
\label{sec:orgdacb8be}
\emph{Package-local nickname} (or \emph{local nickname}) has the same effects as a usual
\emph{package nickname} (later \emph{global nickname}), except that these effects only apply
when \texttt{*package*} is bound to a package for which the nickname has been defined.

That means that calls to \texttt{find-package} with a \emph{local nickname} defined in the
\emph{current package} should return the package nicknamed by this nickname.

This also affects all implied calls to \texttt{find-package}, including those performed by
the lisp reader.

In addition, to maintain \textbf{print-read} consistency, the lisp printer is affected by
\emph{local nicknames} defined in the \emph{current package}.
For details see \hyperref[sec:org335fb35]{Issue 2}.

\emph{Local nickname} is allowed to shadow a \emph{package name} or a \emph{global nickname},
except for the names \texttt{\#:CL}, \texttt{\#:COMMON-LISP} and \texttt{\#:KEYWORD} which should always
refer to their packages.

The consequences of adding \emph{local nicknames} to the packages \texttt{\#:COMMON-LISP} and
\texttt{\#:KEYWORD} are undefined.
\subsection{API}
\label{sec:org6aa01c8}
\subsubsection{defpackage}
\label{sec:org2302211}
\texttt{defpackage} options are extended to include \emph{local-nicknames-option}:
\begin{verbatim}
local-nicknames-option ::= (:local-nicknames (nickname package)*)
\end{verbatim}


Each pair specifies a \emph{local nickname} \texttt{nickname} for the corresponding \texttt{package}.

This option may appear more than once.
\begin{enumerate}
\item Arguments and Values:
\label{sec:org92ba772}
\texttt{nickname} must be a \emph{string designator}.

\texttt{package} must be a \emph{package designator}.
\item Exceptional situations
\label{sec:orgd90db0b}
An error of type \texttt{package-error} is signaled when a package designated by
\texttt{package} does not exists.

Name conflict errors are handled by the underlying calls to
\texttt{add-package-local-nickname}.

See \hyperref[sec:org07f12c7]{add-package-local-nickname: exceptional situations}.
\item Implementation dependent
\label{sec:org4ca0aac}
The behaviour is unspecified when a \emph{local nickname} is specified for the package
that is being defined. (See \hyperref[sec:orga574aea]{Issue 4}.)

The behaviour is unspecified when supplied \emph{local nicknames} are at variance with
the current state of the package that is being defined. An implementation might
choose to remove all present \emph{local nicknames} at the begining of each
redefinition of the package.
\end{enumerate}
\subsubsection{make-package}
\label{sec:orgd9ef999}
(\textbf{PROPOSAL}: see \hyperref[sec:org28d6b88]{Issue 6}.)

\texttt{make-package} lambda list is extended to include an additional keyword argument
\texttt{:local-nicknames}:
\begin{verbatim}
local-nicknames ::= ((nickname package)*)
\end{verbatim}


\texttt{local-nicknames} defaults to an \emph{empty list}.

\texttt{local-nicknames} must be a \emph{list} each element of which must be a \emph{list} of form
\texttt{(nickname package)}. Specifies \emph{local nicknames} in the new \emph{package}.
\begin{enumerate}
\item Arguments and Values:
\label{sec:org0914f7d}
\texttt{local-nicknames} must be a a \emph{list} of pairs \texttt{(nickname package)}.

\texttt{nickname} must be a \emph{string designator}.

\texttt{package} must be a \emph{package designator}.
\item Exceptional situations
\label{sec:org1f5ffe3}
An error of type \texttt{package-error} is signaled when a package designated by
\texttt{package} does not exists.

Name conflict errors are handled by the underlying calls to
\texttt{add-package-local-nickname}.

See \hyperref[sec:org07f12c7]{add-package-local-nickname: exceptional situations}.
\item Implementation dependent
\label{sec:org660f9c3}
The behaviour is unspecified when a \emph{local nickname} is specified for the package
that is being defined. (See \hyperref[sec:orga574aea]{Issue 4}.)
\end{enumerate}
\subsubsection{add-package-local-nickname}
\label{sec:org1dcc074}
\begin{verbatim}
(add-package-local-nickname nickname actual-package &optional designated-package)
  => designated-package-object
\end{verbatim}

\texttt{designated-package} defaults to the \emph{current package}.

Adds a \emph{package-local nickname} \texttt{nickname} for the \texttt{actual-package} in the
\texttt{designated-package}.

Returns the package designated by \texttt{designated-package}. (But also see \hyperref[sec:org6f1e9ed]{Issue 1}.)

If a \emph{nickname} is already defined, checks that it is defined for the package
designated by \texttt{actual-package}.
\begin{enumerate}
\item Arguments and Values
\label{sec:org7b3d17b}
\texttt{nickname} must be a \emph{string designator}.

\texttt{actual-package} and \texttt{designated-package} must be \emph{package designators}.

\texttt{designated-package-object} is of type \emph{package}.
\item Exceptional situations
\label{sec:org07f12c7}
If a package designated by \texttt{actual-package} or a package designated by
\texttt{designated-package} does not exists, an error of type \emph{package-error}
must be signaled.

If \texttt{nickname} is one of the names \texttt{\#:CL}, \texttt{\#:COMMON-LISP} or \texttt{\#:KEYWORD},
an error of type \emph{package-error} must be signaled.

If \texttt{nickname} is a \emph{local nickname} for a package different from
\texttt{actual-package}, an error of type \emph{package-error} must be signaled.
\item Implementation dependent
\label{sec:org1018bc3}
The consequences are undefined when \texttt{designated-package} designates one of the
packages \texttt{\#:COMMON-LISP} or \texttt{\#:KEYWORD}.

(\textbf{PROPOSAL}: see \hyperref[sec:org39f8c46]{Issue 5}.)

If \texttt{nickname} shadows the \texttt{designated-package}'s \emph{package name} or one of its
\emph{global nicknames}, a style warning might signaled.
\end{enumerate}
\subsubsection{remove-package-local-nickname}
\label{sec:org4664ddd}
\begin{verbatim}
(remove-package-local-nickname old-nickname &optional designated-package)
  => nickname-removed-p
\end{verbatim}

\texttt{designated-package} defaults to the \emph{current package}.

If \texttt{designated-package} has \texttt{old-nickname} as a \emph{local nickname}, it is removed.

Returns \emph{true} if the \texttt{old-nickname} existed and was removed, and \texttt{NIL}
otherwise. (But also see \hyperref[sec:org6f1e9ed]{Issue 1}.)
\begin{enumerate}
\item Arguments and Values
\label{sec:orgb74602c}
\texttt{old-nickname} must be a \emph{string designator}.

\texttt{designated-package} must be a \emph{package designator}.

\texttt{nickname-removed-p} is a \emph{generalized boolean}.
\item Exceptional situations
\label{sec:orgaeb0739}
If a package designated by \texttt{designated-package} does not exists, an error of type
\emph{package-error} must be signaled.
\end{enumerate}
\subsubsection{package-local-nicknames}
\label{sec:org2747466}
\begin{verbatim}
(package-local-nicknames package)
  => local-nicknames-alist
local-nicknames-alist ::= ((nickname . package)*)
\end{verbatim}

Returns an \emph{alist} describing \emph{local nicknames} defined in the package designated
by \texttt{package}.
\begin{enumerate}
\item Arguments and Values
\label{sec:orgcb97b63}
\texttt{package} must be a \emph{package designator}.

\texttt{local-nicknames-alist} is an \emph{alist} with keys of type \emph{string} and values of
type \emph{package}.

\texttt{nickname} must be a \emph{string}.

\texttt{package} must be a \emph{package}.
\item Exceptional situations
\label{sec:org76f71ed}
An error of type \texttt{package-error} is signaled when a package designated by
\texttt{package} does not exists.
\item Notes
\label{sec:org31a80ba}
The returned \emph{alist} must be safe to be modified by the user.
\end{enumerate}
\subsubsection{package-locally-nicknamed-by-list}
\label{sec:org45f0e67}
\begin{verbatim}
(package-locally-nicknamed-by-list package)
  => packages-list
\end{verbatim}

Returns a \emph{list} of packages that have a \emph{local nickname} defined for the package
designated by \texttt{package}.
\begin{enumerate}
\item Arguments and Values
\label{sec:org89904ea}
\texttt{package} must be a \emph{package designator}.

\texttt{packages-list} is a \emph{list} with elements of type \emph{package}.
\item Exceptional situations
\label{sec:org25e3178}
An error of type \texttt{package-error} is signaled when a package designated by
\texttt{package} does not exists.
\item Notes
\label{sec:org4fe67bd}
The returned \emph{list} must be safe to be modified by the user.
\end{enumerate}
\subsection{Affected symbols}
\label{sec:orgaf51aa9}
\subsubsection{defpackage}
\label{sec:org186f780}
See \hyperref[sec:org2302211]{defpackage}.
\subsubsection{make-package}
\label{sec:org9ec1a24}
See \hyperref[sec:orgd9ef999]{make-package}.
\subsubsection{find-package}
\label{sec:orgaaf0775}
When argument to \texttt{find-package} is a \emph{local nickname} that is defined in the
\emph{current package}, it returns the package named by this nickname.

This also affects all implied calls to \texttt{find-package}, including but not limited
to those performed by the lisp reader as well as those performed by \texttt{export},
\texttt{find-symbol}, \texttt{import}, \texttt{rename-package}, \texttt{shadow}, \texttt{shadowing-import},
\texttt{delete-package}, \texttt{with-package-iterator}, \texttt{unexport}, \texttt{unintern}, \texttt{in-package},
\texttt{unuse-package}, \texttt{use-package}, \texttt{do-symbols}, \texttt{do-external-symbols},
\texttt{do-all-symbols}, \texttt{intern}, \texttt{package-name}, \texttt{package-nicknames},
\texttt{package-shadowing-symbols}, \texttt{package-use-list}, \texttt{package-used-by-list}.

\texttt{add-package-local-nickname}, \texttt{remove-package-local-nickname},
\texttt{package-local-nicknames} and \texttt{package-locally-nicknamed-by} are also affected.

(\textbf{PROPOSAL}: see \hyperref[sec:orgf59d09e]{Issue 8}.)

The only exception is the \texttt{format}'s \emph{tilde slash} directive, which should \textbf{not}
use \emph{local nicknames} of any package when looking up the symbol specified.
\subsubsection{rename-package}
\label{sec:org671adcf}
When a package is renamed with \texttt{rename-package} it maintains all \emph{local nicknames}
it is nicknamed by, as well as all \emph{local nicknames} it has defined.
\begin{enumerate}
\item Implementation dependent
\label{sec:orgb2ead79}
(\textbf{PROPOSAL}: see \hyperref[sec:org39f8c46]{Issue 5}.)

If the \emph{new-name} or one of the \emph{new-nicknames} is shadowed by one of the \emph{local
nicknames} of the package being redefined, a warning might be signaled.
\end{enumerate}
\subsubsection{delete-package}
\label{sec:org379d2ff}
When a package is deleted with \texttt{delete-package} all \emph{local nicknames} defined in
other packages that it was nicknamed by must be removed, as well as all \emph{local
nicknames} defined in the package that is being deleted.

This also means that a deleted package must not be available via calls to
\texttt{package-locally-nicknamed-by-list} and \texttt{package-local-nicknames}.
\subsubsection{format}
\label{sec:orgdf53675}
See \hyperref[sec:orgf59d09e]{Issue 8}.
\subsubsection{$\backslash$*features$\backslash$*}
\label{sec:orgbb01881}
If an implementation supports package-local nicknames it should add symbols
\texttt{:package-local-nicknames} and \texttt{:cdr-15} (per CDR 14) to \texttt{*features*}.
\subsection{Examples}
\label{sec:org64da27f}
[TODO]
\section{ISSUES}
\label{sec:org5ba2b5f}
\href{./issues/1.org}{Issue 1 ([ADD-/REMOVE-]PACKAGE-LOCAL-NICKNAME return values)}
\href{./issues/2.org}{Issue 2 (PRINT-READ consistency)}
\href{./issues/3.org}{Issue 3 (Local nicknames effect on DEFPACKAGE, MAKE-PACKAGE and others)}
\href{./issues/4.org}{Issue 4 (Local nicknames of the package being defined)}
\href{./issues/5.org}{Issue 5 (Local nickname shadowing package's own name)}
\href{./issues/6.org}{Issue 6 (Additional keyword argument to MAKE-PACKAGE)}
\href{./issues/7.org}{Issue 7 (Multiple local nicknames)}
\href{./issues/8.org}{Issue 8 (Interaction with FORMAT)}
\href{./issues/9.org}{Issue 9 (Empty package local name)}

\subsection{Issue 1 ([ADD-/REMOVE-]PACKAGE-LOCAL-NICKNAME return values)}
\label{sec:org6f1e9ed}
\subsubsection{Description}
\label{sec:org71cada1}
Return values of \texttt{add-package-local-nickname} and \texttt{remove-package-local-nickname}
are inconsistent. The first one always returns \emph{designated package}, while the
second one returns \emph{true} (generalized boolean) if a nickname was removed.

Moreover, there is no consensus among current implementations as to what the second
function (\texttt{remove-package-local-nickname}) should return.
\subsubsection{Examples}
\label{sec:org02f1b6f}
\begin{verbatim}
(defpackage #:foo (:use))
(defpackage #:bar (:use))

(add-package-local-nickname '#:nick '#:bar '#:foo)
; => #<PACKAGE "FOO">  (sbcl, ccl, ecl, acl, abcl, clasp, lispworks)
(add-package-local-nickname '#:nick '#:bar '#:foo)
; => #<PACKAGE "FOO">  (sbcl, ccl, ecl, acl, abcl, clasp, lispworks)
(remove-package-local-nickname '#:nick '#:foo)
; => T  (sbcl, ccl, ecl, acl, clasp)
; => #<PACKAGE "BAR">  (abcl)
; => NIL  (lispworks)
(remove-package-local-nickname '#:nick '#:foo)
; => NIL  (sbcl, ccl, ecl, acl, abcl, clasp, lispworks)
\end{verbatim}
\subsubsection{Current behavior}
\label{sec:org1a1c99c}
sbcl, ccl, ecl, acl, abcl, clasp, lispworks:
  \texttt{add-package-local-nickname} always returns \emph{designated package}.

sbcl, ccl, ecl, acl, clasp:
  \texttt{remove-package-local-nickname} returns \texttt{T} if a nickname was removed,
  and \texttt{NIL} otherwise.

abcl:
  \texttt{remove-package-local-nickname} if a nickname was removed, returns \emph{true} (more
  specifically - the package that was nicknamed by that nickname), and \texttt{NIL}
  otherwise.

lispworks:
  \texttt{remove-package-local-nickname} always returns \texttt{NIL}.
\subsubsection{Proposal DESIGNATED-PACKAGE-IF-SUCCESSFUL}
\label{sec:org80fc760}
\begin{itemize}
\item \texttt{add-package-local-nickname} should return \emph{designated package} if a new nickname
was added and \texttt{NIL} otherwise, if the nickname already existed.
\item \texttt{remove-package-local-nickname} should return \emph{designated package} if a nickname
was removed and \texttt{NIL} otherwise.
\end{itemize}
\subsubsection{Proposal ALWAYS-T}
\label{sec:org6999275}
Similar to \texttt{use-package} and \texttt{unuse-package}:
\begin{itemize}
\item \texttt{add-package-local-nickname} should always return \texttt{T}.
\item \texttt{remove-package-local-nickname} should always return \texttt{T}.
\end{itemize}
\subsubsection{Proposal ELIMINATE-GENERALIZED\textsubscript{BOOLEAN}}
\label{sec:org59d96de}
\begin{itemize}
\item \texttt{add-package-local-nickname} should alwyas return \emph{designated package}.
\item \texttt{remove-package-local-nickname} should return \texttt{T} if a nickname was removed and
\texttt{NIL} otherwise.
\end{itemize}

\subsection{Issue 2 (PRINT-READ consistency)}
\label{sec:org335fb35}
\subsubsection{Description}
\label{sec:orgb11369c}
Lisp reader uses \texttt{find-package} when reading a symbol, which is affected by the
\emph{local nicknames} of the \emph{current package}. That means that to maintain \textbf{print-read}
consistency when printing a symbol, a good \emph{package prefix} must be used - such that
calling \texttt{find-package} on it in the \emph{current package} returns the symbol's \emph{home
package}.

There are several situations to consider:
\begin{enumerate}
\item Symbol is \emph{apparently uninterned}.

\emph{In this case it must be printed without any package prefix, preceeded by \texttt{\#:}.}

\item Symbol is accessible in the \emph{current package}.

\emph{In this case it must be printed without any package prefix.}

\item Symbol's \emph{home package} \emph{name} or one of its \emph{global nicknames} is not shadowed
by any \emph{local nickname} defined in the \emph{current package}.

\emph{In this case that name or global nickname can be used as the package prefix.}

\item There \textbf{exists} a \emph{local nickname} defined in the \emph{current package} for the
symbol's \emph{home package}.

\emph{In this case that local nickname can be used as the package prefix.}

\item Symbol's \emph{home package} \emph{name} and all of its \emph{global nicknames} are shadowed by
the \emph{local nicknames} of the \emph{current package} and there \textbf{is no} \emph{local nickname}
defined in the \emph{current package} for the symbol's \emph{home package}.

\emph{It is not clear how the symbol must be printed, see PROPOSALS.}
\end{enumerate}
\subsubsection{Examples}
\label{sec:org3ae9387}
\begin{verbatim}
(defpackage #:foo
  (:use)
  (:export #:+))

(defpackage #:bar
  (:use #:cl)
  (:local-nicknames (#:foo #:cl)))

(let ((*package* (find-package '#:bar)))
  (print 'foo:+))
; >> FOO:+  (sbcl, ccl, ecl, acl, abcl, clasp, lispworks)

;; In the package #:BAR symbol FOO:+ refers to CL:+
\end{verbatim}

\begin{verbatim}
(defpackage #:foo-a (:use) (:export #:quux))
(defpackage #:foo-b (:use) (:export #:quux))

(defpackage #:bar
  (:use)
  (:local-nicknames (#:foo-a #:foo-b)
                    (#:foo-b #:foo-a)))

(let ((*package* (find-package '#:bar)))
  (print 'foo-a:quux))
; >> FOO-B:QUUX  (sbcl, ccl, abcl, lispworks)
; >> FOO-A:QUUX  (ecl, acl, clasp)

;; In the package #:BAR symbol FOO-A:QUUX refers to FOO-B:QUUX
\end{verbatim}
\subsubsection{Current behavior}
\label{sec:org8cd9e7a}
sbcl, ccl, abcl, lispworks:
  When exists in the \emph{current package}, a \emph{local nickname} is used as a package
  prefix when printing a symbol.

ecl, acl, clasp:
  \emph{local nickname} is never used as a package prefix when printing a symbol.
\subsubsection{Proposal SHARPSIGN-DOT}
\label{sec:org90675cb}
In the case (5) the symbol must be printed using the \texttt{\#.} syntax:

\begin{verbatim}
#.(cl:let ((cl:*package* (cl:find-package "KEYWORD")))
    (cl:find-symbol "BAR" "FOO"))
;; or
#.(cl:let ((cl:*package* (cl:find-package "KEYWORD")))
    (cl:intern "BAR" "FOO"))
\end{verbatim}

Note that \texttt{\#:KEYWORD} name is reserved for the \texttt{\#:KEYWORD} package and
cannot be used as a \emph{local nickname} thus this expression will always
evaluate to the symbol \texttt{foo::bar}.

If \texttt{*read-eval*} is \emph{false} and \texttt{*print-readably*} is \emph{true} an error of type
\texttt{print-not-readable} must be signalled.
\subsubsection{Proposal SHARPSIGN-COLON}
\label{sec:org3e0a44b}
In case (5) the symbol should be printed using the \emph{extended \texttt{\#:} syntax}:
\begin{verbatim}
#:(package name)
#::(package name)
\end{verbatim}

\emph{Shinmera's idea}.
\subsubsection{Proposal SHARPSIGN-BACKQUOTE}
\label{sec:org7a2e477}
In case (5) the symbol must be printed using the new \texttt{\#`} syntax for reading an
expression ignoring \emph{local nicknames} in the \emph{current package}:
\begin{verbatim}
#`foo:bar
#`foo::bar
\end{verbatim}

It can be implemented roughly as follows:
\begin{verbatim}
(defun |#`-reader| (stream subchar arg)
  (declare (ignore subchar arg))
  (let* ((current-package *package*)
         (local-nicknames (package-local-nicknames current-package)))
    (loop for (nick . package) in local-nicknames
          do (remove-package-local-nickname nick current-package))
    (unwind-protect
      (read stream t nil t)
      (loop for (nick . package) in local-nicknames
            do (add-package-local-nickname nick package current-package)))))

(set-dispatch-macro-character #\# #\` #'|#`-reader|)
\end{verbatim}
It is \emph{implementation-dependent} whether \emph{local nicknames} are actually removed
from the \emph{current package} or not.
\subsubsection{Proposal PRINT-UNREADABLY}
\label{sec:org1f6297c}
In the case (5) the symbol must be printed unreadably using the \texttt{\#<} syntax:
\begin{verbatim}
#<SYMBOL IN THE SHADOWED PACKAGE FOO:BAR>
#<SYMBOL IN THE SHADOWED PACKAGE FOO::BAR>
\end{verbatim}

(Specifics are \emph{implementation-dependent}.)

If \texttt{*print-readably*} is \emph{true}, an error of type \texttt{print-not-readable} must be
signalled.
\subsubsection{Proposal THREE-FOUR-PACKAGE-MARKERS}
\label{sec:org71c7fa5}
In the case (5) the symbol must be printed using \texttt{:::} and \texttt{::::} syntax as follows:
\begin{verbatim}
foo:::bar   ; same as (cl:find-symbol "BAR" "FOO") in the #:KEYWORD package
foo::::bar  ; same as (cl:intern "BAR" "FOO") in #:KEYWORD package
\end{verbatim}
\subsubsection{Links}
\label{sec:orgdb355c0}
See \href{https://www.lispworks.com/documentation/HyperSpec/Body/22\_acca.htm}{CLHS 22.1.3.3.1 Package Prefixes for Symbols}.

\subsection{Issue 3 (Local nicknames effect on DEFPACKAGE, MAKE-PACKAGE and others)}
\label{sec:org7ab08ff}
\subsubsection{Description}
\label{sec:orgc334ff7}
It is not clear whether \emph{local nicknames} of the \emph{current package} should affect
the resolution of package designators provided in \texttt{make-package} and \texttt{defpackage},
as well as other functions and macros taking a package designator as an argument
(\texttt{in-package}, \texttt{add-package-local-nickname} and others).
\subsubsection{Examples}
\label{sec:org665efbb}
\begin{verbatim}
(defpackage #:foo-a (:use) (:export #:x))
(defpackage #:foo-b (:use) (:export #:x))

(defpackage #:bar
  (:use #:cl)
  (:local-nicknames (#:foo-a #:foo-b)
                    (#:foo-b #:foo-a)))

(in-package #:bar)

(defpackage #:quux-1
  (:use #:foo-a))
(package-name (symbol-package 'quux-1::x))
; => "FOO-B"  (sbcl, ccl, acl, abcl, lispworks)
; => "FOO-A"  (ecl, clasp)

(make-package '#:quux-2 :use '(#:foo-a))
(package-name (symbol-package 'quux-2::x))
; => "FOO-B"  (sbcl, ccl, ecl, acl, abcl, clasp, lispworks)

(defpackage #:quux-3
  (:use)
  (:local-nicknames (#:foo #:foo-a)))
(let ((*package* (find-package '#:quux-3)))
  (package-name (find-package '#:foo)))
; => "FOO-B"  (ccl, ecl, acl, abcl)
; => "FOO-A"  (sbcl, clasp, lispworks)

(import (car (find-all-symbols (string '#:add-package-local-nickname))))
(defpackage #:quux-4
  (:use))
(add-package-local-nickname '#:foo '#:foo-a '#:quux-4)
(let ((*package* (find-package '#:quux-4)))
  (package-name (find-package '#:foo)))
; => "FOO-B"  (ccl, ecl, clasp, abcl, lispworks)
; => "FOO-A"  (sbcl)

(use-package '#:foo-a '#:quux-4)
(package-name (symbol-package 'quux-4::x))
; => "FOO-B"  (sbcl, ccl, ecl, clasp, abcl, lispworks)
\end{verbatim}
\subsubsection{Current behavior}
\label{sec:org89b17bd}
sbcl, lispworks:
  only \texttt{:local-nicknames} clause is \textbf{not} affected by \emph{local nicknames}.

ccl, acl, abcl:
  all options are affected.

ecl:
  only the \texttt{:local-nicknames} clause and keyword arguments (\texttt{:use} and
  \texttt{:local-nicknames}) are affected.

clasp:
  only keyword argument \texttt{:use} is affected.

Known exceptions in other functions/macros:
sbcl: \texttt{add-package-local-nickname} is not affected.
lispworks: \texttt{in-package} is not affected.
\subsubsection{Proposal ALL-AFFECTED}
\label{sec:org2aaf2e4}
All \texttt{defpackage} clauses (\texttt{:use}, \texttt{:local-nicknames}, \texttt{:import-from},
\texttt{:shadowing-import-from}) as well as all keyword arguments to \texttt{make-package}
(\texttt{:use} and \texttt{:local-nicknames}) must be affected by the \emph{local nicknames} of the
\emph{current package}.

All functions and macros taking a package designator as an argument must be
affected as well.

A non-exhaustive list of affected functions and macros:
  \texttt{export}, \texttt{find-symbol}, \texttt{import}, \texttt{rename-package}, \texttt{shadow},
  \texttt{shadowing-import}, \texttt{delete-package}, \texttt{with-package-iterator}, \texttt{unexport},
  \texttt{unintern}, \texttt{in-package}, \texttt{unuse-package}, \texttt{use-package}, \texttt{do-symbols},
  \texttt{do-external-symbols}, \texttt{do-all-symbols}, \texttt{intern}, \texttt{package-name},
  \texttt{package-nicknames}, \texttt{package-shadowing-symbols}, \texttt{package-use-list},
  \texttt{package-used-by-list}, \texttt{add-package-local-nickname},
  \texttt{remove-package-local-nickname}, \texttt{package-local-nicknames},
  \texttt{package-locally-nicknamed-by}.

\subsection{Issue 4 (Local nicknames of the package being defined)}
\label{sec:orga574aea}
\subsubsection{Description}
\label{sec:org0699eb9}
It is not clear whether \emph{local nicknames} of the package \textbf{being defined} should
affect \texttt{make-package} or \texttt{defpackage}.
\subsubsection{Examples}
\label{sec:orga30610b}
\begin{verbatim}
(defpackage #:foo-a (:use) (:export #:x))
(defpackage #:foo-b (:use) (:export #:x))

(defpackage #:bar
  (:local-nicknames (#:foo-a #:foo-b)
                    (#:foo-b #:foo-a))
  (:use #:foo-a))

(package-name (symbol-package 'bar::x))
; => "FOO-A"  (sbcl, ccl, acl, abcl, clasp, lispworks)
; => "FOO-B"  (ecl)
\end{verbatim}
\subsubsection{Current behavior}
\label{sec:org03f228b}
sbcl, ccl, acl, abcl, lispworks: not affected

ecl: \texttt{:use}, \texttt{:import-from} and \texttt{:shadowing-import-from} are affected.

clasp: \texttt{:local-nicknames} is affected by previous \texttt{:local-nicknames} clauses.
\subsubsection{Proposal NO-EFFECT}
\label{sec:org767def9}
Local nicknames of a package being defined should not affect other defpackage
clauses (\texttt{:use}, \texttt{:local-nicknames}, \texttt{:import-from}, \texttt{:shadowing-import-from}).

The keyword argument \texttt{:local-nicknames} to \texttt{make-package} should not affect the
\texttt{:use} keyword argument either.

\subsection{Issue 5 (Local nickname shadowing package's own name)}
\label{sec:org39f8c46}
\subsubsection{Description}
\label{sec:org335ffd4}
It is not clear whether it is valid to have a \emph{local nickname} in a package
shadowing its own name or nickname.
\subsubsection{Examples}
\label{sec:org19facbd}
\begin{verbatim}
(defpackage #:foo
  (:use)
  (:nicknames #:bar)
  (:local-nicknames (#:foo #:cl)
                    (#:bar #:cl)))
; => continuable error  (sbcl, ccl, abcl)
; => error  (lispworks)
; => ok  (ecl, acl, clasp)
\end{verbatim}
\subsubsection{Current behavior}
\label{sec:org5b3edef}
sbcl, ccl, abcl and lispwork signal an error.
\subsubsection{Proposal ALLOW}
\label{sec:orgf5c5dc7}
It should be allowed to use package's own name or global nickname, but a
style-warning can be signalled.

\subsection{Issue 6 (Additional keyword argument to MAKE-PACKAGE)}
\label{sec:org28d6b88}
\subsubsection{Current behavior}
\label{sec:org99a7e23}
sbcl, ccl, abcl, clasp, lispworks: no additional key argument.

ecl: has an additional keyword argument \texttt{:local-nicknames}, but it is undocumented
and it segfaults on incorrect usage. The expected value is a list of conses:
\texttt{((nickname . package)*)}.

acl: has an additional keyword argument \texttt{:local-nicknames}. The expected value is
a list of lists: \texttt{((nickname package)*)}.
\subsubsection{Proposal EXTRA-KEYWORD-ARGUMENT}
\label{sec:org5473b89}
Add \texttt{:local-nicknames} keyword argument to \texttt{make-package}:
\begin{verbatim}
local-nicknames ::= ((nickname package)*)
\end{verbatim}

\texttt{nickname} must be a \emph{string designator}.
\texttt{package} must be a \emph{package designator}.

\texttt{local-nicknames} defaults to an \emph{empty list}.

See \hyperref[sec:orgd9ef999]{make-package}.

\subsection{Issue 7 (Multiple local nicknames)}
\label{sec:orge0c95b8}
\subsubsection{Description}
\label{sec:orgb684821}
It is not clear whether \texttt{package-locally-nicknamed-by-list} should be allowed to return
lists with duplicate entries (when there are multiple local nicknames in one package).
\subsubsection{Examples}
\label{sec:org117d806}
\begin{verbatim}
(defpackage #:foo
  (:use)
  (:local-nicknames (#:bar #:cl)
                    (#:baz #:cl)))
(mapcar #'package-name (package-locally-nicknamed-by-list '#:cl))
; => ("FOO")  (sbcl, acl, clasp, lispworks)
; => ("FOO" "FOO")  (ccl, ecl, abcl)
\end{verbatim}
\subsubsection{Current behavior}
\label{sec:org71d44ce}
sbcl, acl, clasp, lispworks:
  \texttt{package-locally-nicknamed-by-list} never contains duplicate entries.

ccl, abcl, ecl:
  \texttt{package-locally-nicknamed-by-list} might contain duplicate entries.
\subsubsection{Proposal NO-DUPLICATES}
\label{sec:org33fb4f8}
\texttt{package-locally-nicknamed-by-list} must return a list without duplicate entries.

\subsection{Issue 8 (Interaction with FORMAT)}
\label{sec:orgf59d09e}
\emph{by |3b|}
\subsubsection{Description}
\label{sec:org652fe55}
It is not clear how \emph{local nicknames} should affect \texttt{format}'s \texttt{\textbackslash{}\textasciitilde{}//} directive.

First, it would be inconvinient if it wouldn't be possible to use \emph{local
nicknames} with the \texttt{\textbackslash{}\textasciitilde{}//} directive.

Secondly, it would be unintuitive if the function used would depend on the
\emph{current package} at the \textbf{execution} time. This also might break [existing] code,
if a local nickname in the \emph{current package} shadows a package that contains a
function used by a control string in another function.

Finally, if the call to \texttt{format} is not compiled, it is hard to impossible to find
the function using \emph{local nicknames} of the package that was the \emph{current package}
at the \textbf{compile} time.
\subsubsection{Examples}
\label{sec:orgb3b3786}
\begin{verbatim}
(defpackage #:foo-a (:use) (:export #:ff))
(defpackage #:foo-b (:use) (:export #:ff))

(defun foo-a:ff (stream &rest args)
  (declare (ignore args))
  (format stream "FOO-A:FF"))

(defun foo-b:ff (stream &rest args)
  (declare (ignore args))
  (format stream "FOO-B:FF"))

(defpackage #:bar-a
  (:use #:cl)
  (:local-nicknames (#:nick #:foo-a)))

(defpackage #:bar-b
  (:use #:cl)
  (:local-nicknames (#:nick #:foo-b)))

(in-package #:bar-a)

(defun test ()
  (format t "Called ~/nick:ff/ & " nil)
  (let ((*package* (find-package (quote #:bar-a))))  ; or #.*package*
    (format t "~/nick:ff/~%" nil)))

(test)
; => "Called FOO-A:FF & FOO-A:FF"  (sbcl, ccl, ecl, acl, abcl, clasp)
; lispworks errors (NICK package not found)
(let ((*package* (find-package (quote #:bar-b))))
  (test))
; => "Called FOO-A:FF & FOO-A:FF"  (sbcl, clasp)
; => "Called FOO-B:FF & FOO-A:FF"  (ccl, ecl, acl, abcl)
; lispworks errors (NICK package not found)
\end{verbatim}
\subsubsection{Proposal NO-LOCAL-NICKNAMES}
\label{sec:org7425cd7}
\texttt{format}'s \texttt{\textbackslash{}\textasciitilde{}//} directive must \textbf{not} use \emph{local nicknames} of any package when
looking up the specified symbol.

Rationale: In the spirit of how when the package is not specified, the symbol is
not looked up in the \emph{current package}, but instead in the \texttt{\#:CL-USER} package;
the \emph{tilde slash} directive should not depend on the value of \texttt{*package*} at any
time.  Specifying it to use \emph{local nicknames} of the \texttt{\#:CL-USER} package instead
would risk breaking the existing code when adding local nicknames to that package.
\subsubsection{Links}
\label{sec:org44c87b6}
See \href{https://www.lispworks.com/documentation/HyperSpec/Body/22\_ced.htm}{CLHS 22.3.5.4 Tilde Slash: Call Function}.

\subsection{Issue 9 (Empty package local name)}
\label{sec:org2b60e2e}
\subsubsection{Description}
\label{sec:org46bc3c3}
It is not clear whether it should be allowed to use \texttt{""} as a local nickname,
and in the case this is allowed, whether it should affect the \texttt{:xxxx} syntax.
\subsubsection{Examples}
\label{sec:orgd848ea7}
\begin{verbatim}
(defpackage #:foo
  (:use #:cl)
  (:local-nicknames ("" #:cl)))

(in-package #:foo)

(package-name (symbol-package ':*package*))
; => "KEYWORD"  (sbcl, ccl, ecl, abcl, clasp, lispworks)
; => "COMMON-LISP"  (acl)

(package-name (symbol-package '||:*package*))
; => "KEYWORD"  (ecl, clasp, lispworks)
; => "COMMON-LISP"  (sbcl, ccl, acl)
; abcl errors
\end{verbatim}
\subsubsection{Current behavior}
\label{sec:org9043d01}
sbcl, ccl:
\texttt{:xxxx} is read as a keyword;
\texttt{||:xxxx} is read as a symbol in the package named or nicknamed \texttt{""}.

ecl, clasp, lispworks:
\texttt{||:xxxx} and \texttt{:xxxx} are read as a keyword.

acl:
\texttt{:xxxx} and \texttt{||:xxxx} are read as a symbol in the package named or nicknamed \texttt{""}.
\texttt{""} is by default a global nickname for the \texttt{\#:KEYWORD} package.

abcl:
\texttt{:xxxx} is read as a keyword;
\texttt{||:xxxx} syntax cannot be read (attemts result in an error).
\subsubsection{Proposal ALLOW-BUT-KEEP-KEYWORDS}
\label{sec:orga313bb7}
The \texttt{""} local nickname should be explicitely allowed. \texttt{:xxxx} should be always
read as a keyword regardless of package names or nicknames. \texttt{||:xxxx} should be
read as a symbol in the package named or nicknamed by \texttt{""}.
\subsubsection{Proposal ALLOW-FUN}
\label{sec:org70bd308}
The \texttt{""} local nickname should be explicitely allowed. Both \texttt{:xxxx} and \texttt{||:xxxx}
should be read as a symbol in the package named or nicknamed by \texttt{""}.
\subsubsection{Links}
\label{sec:org660fe67}
See \href{https://github.com/s-expressionists/wscl/issues/63}{WSCL issue 63}.
\section{Links}
\label{sec:org7f37578}
3b's \href{https://github.com/3b/package-local-nicknames/blob/master/docs.org}{notes} on package-local nicknames.

phoe's \href{https://github.com/phoe/trivial-package-local-nicknames}{tests}.

SBCL's \href{https://www.sbcl.org/manual/\#Package\_002dLocal-Nicknames}{manual entry}.

Section 4.3 of the ABCL's manual. (\href{https://github.com/armedbear/abcl/blob/master/doc/manual/abcl.tex\#L1249}{\TeX{} file on github})
\section{Copying and License}
\label{sec:orgc258bbc}
[TODO]
\end{document}