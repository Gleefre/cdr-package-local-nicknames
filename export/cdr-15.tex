% Created 2024-07-06 Sat 13:22
% Intended LaTeX compiler: pdflatex
\documentclass[11pt]{article}
\usepackage[utf8]{inputenc}
\usepackage[T1]{fontenc}
\usepackage{graphicx}
\usepackage{longtable}
\usepackage{wrapfig}
\usepackage{rotating}
\usepackage[normalem]{ulem}
\usepackage{amsmath}
\usepackage{amssymb}
\usepackage{capt-of}
\usepackage{hyperref}
\usepackage[margin=1in]{geometry}
\usepackage[margin=1in]{geometry}
\usepackage[margin=1in]{geometry}
\usepackage[margin=1in]{geometry}
\usepackage[margin=1in]{geometry}
\usepackage[margin=1in]{geometry}
\usepackage[margin=1in]{geometry}
\usepackage[margin=1in]{geometry}
\usepackage[margin=1in]{geometry}
\usepackage[margin=1in]{geometry}
\usepackage[margin=1in]{geometry}
\author{Gleefre}
\date{\today}
\title{Package-Local Nicknames}
\hypersetup{
 pdfauthor={Gleefre},
 pdftitle={Package-Local Nicknames},
 pdfkeywords={},
 pdfsubject={This is a CDR specification for package-local nicknames.},
 pdfcreator={Emacs 28.2 (Org mode 9.5.5)}, 
 pdflang={English}}
\begin{document}

\maketitle
\tableofcontents

[THIS IS A DRAFT]

\section{Specification}
\label{sec:org760256d}

\subsection{Introduction}
\label{sec:orgeaaf178}
This is a specification for the \texttt{Package-Local Nicknames} extension in Common Lisp.
\subsubsection{Rationale}
\label{sec:org6675b74}
Package-local nicknames make it possible to use short and easy-to-use names
without potentially introducing name conflicts as can happen with usual nicknames.
\subsubsection{Current state}
\label{sec:orgbdc9445}
Package-local nicknames are implemented in some form in \texttt{SBCL}, \texttt{CCL}, \texttt{ECL},
\texttt{Clasp}, \texttt{ABCL}, \texttt{Allegro CL}, \texttt{LispWorks}. There is also a pending MR for the
\texttt{CLISP} implementation.

Unfortunately, there are multiple inconsistencies between implementations. All of
them lose \emph{print-read consistency} to some extent, and there are multiple edge
cases that aren't always implemented correctly or in the same way.
\subsubsection{Goal}
\label{sec:org844a30c}
The purpose of this document is to standardize the \texttt{Package-Local Nicknames}
extension and to address some existing issues.

[TODO] This CDR also aims to provide an extensive test suite for the extension.
\subsection{Description}
\label{sec:orgf7569b1}
A \emph{package-local nickname} (or a \emph{local nickname}) defined in some \emph{designated
package} has the same effects as a usual \emph{package nickname} (later referred to as a
\emph{global nickname}), except that these effects only apply when \texttt{*package*} is bound
to that \emph{designated package}.

This means that a call to \texttt{find-package} with a \emph{local nickname} that is defined in
the \emph{current package} returns the package nicknamed by this nickname. This also
affects all implied calls to \texttt{find-package}, including those performed by the Lisp
reader.

In addition, to maintain \emph{print-read consistency}, the Lisp printer is affected by
\emph{local nicknames} defined in the \emph{current package}.  For details see \hyperref[sec:orgf15eb69]{Issue 2}.

A \emph{local nickname} is allowed to shadow a \emph{package name} or a \emph{global nickname},
except for the names \texttt{\#:CL}, \texttt{\#:COMMON-LISP} and \texttt{\#:KEYWORD} which must always
refer to their packages. The consequences of adding \emph{local nicknames} to the
packages \texttt{\#:COMMON-LISP} and \texttt{\#:KEYWORD} are also undefined.
\subsection{API}
\label{sec:org71b17e2}
\subsubsection{defpackage}
\label{sec:orgbd9e3bf}
\begin{enumerate}
\item Description
\label{sec:org419afb8}
The \texttt{defpackage} options are extended to include the \emph{local-nicknames-option}:
\begin{verbatim}
local-nicknames-option ::= (:local-nicknames (nickname package)*)
\end{verbatim}


Each pair specifies a \emph{local nickname} \texttt{nickname} for the corresponding \texttt{package}.

This option may appear more than once.
\item Arguments and Values:
\label{sec:orgfe2f390}
\texttt{nickname} --- a \emph{string designator}.

\texttt{package} --- a \emph{package designator}.
\item Exceptional situations
\label{sec:org365a16e}
An error of type \texttt{package-error} is signaled when a package designated by the
\texttt{package} does not exists.

Name conflict errors are handled by the underlying calls to
\texttt{add-package-local-nickname}.

See \hyperref[sec:orgc29dab7]{add-package-local-nickname: exceptional situations}.
\item Implementation dependent
\label{sec:orgc82ddde}
The consequences are undefined when a \emph{local nickname} is specified for thex
package that is being defined. (See \hyperref[sec:orgd561e6a]{Issue 4}.)

The consequences are undefined when supplied \emph{local nicknames} are at variance
with the current state of the package. An implementation might choose to remove
all existing \emph{local nicknames} at the beginning of each redefinition of the
package.
\end{enumerate}
\subsubsection{make-package}
\label{sec:org1fd55bd}
\begin{enumerate}
\item Description
\label{sec:org4e34509}
(\textbf{Contains proposals}: see \hyperref[sec:orgd3fb702]{Issue 6}.)

The \texttt{make-package} lambda list is extended to include an additional keyword
argument \texttt{:local-nicknames}:
\begin{verbatim}
local-nicknames ::= ((nickname package)*)
\end{verbatim}


\texttt{local-nicknames} specifies zero or more \emph{local nicknames} to be defined in the
new \emph{package}.
\item Arguments and Values:
\label{sec:org3e646a5}
\texttt{local-nicknames} --- a \emph{list} of pairs of form \texttt{(nickname package)}.
The default is an \emph{empty list}.

\texttt{nickname} --- a \emph{string designator}.

\texttt{package} --- a \emph{package designator}.
\item Exceptional situations
\label{sec:org50da4f2}
An error of type \texttt{package-error} is signaled when a package designated by the
\texttt{package} does not exist.

Name conflict errors are handled by the underlying calls to
\texttt{add-package-local-nickname}.

See \hyperref[sec:orgc29dab7]{add-package-local-nickname: exceptional situations}.
\item Implementation dependent
\label{sec:org408aefd}
The consequences are undefined when a \emph{local nickname} is specified for the
package that is being defined. (See \hyperref[sec:orgd561e6a]{Issue 4}.)
\end{enumerate}
\subsubsection{add-package-local-nickname}
\label{sec:orgac7008d}
\begin{verbatim}
(add-package-local-nickname nickname actual-package &optional designated-package)
  => designated-package-object
\end{verbatim}
\begin{enumerate}
\item Arguments and Values
\label{sec:orgc86149f}
\texttt{nickname} --- a \emph{string designator}.

\texttt{actual-package} --- a \emph{package designator}.

\texttt{designated-package} --- a \emph{package designator}.
The default is the \emph{current package}.

\texttt{designated-package-object} --- a \emph{package}.
\item Description
\label{sec:org673b353}
Defines a \emph{package-local nickname} \texttt{nickname} for the \texttt{actual-package} in the
\texttt{designated-package}.

[Also see \hyperref[sec:orgede3d6c]{Issue 1}.] Returns the package designated by the \texttt{designated-package}.

If the \emph{nickname} is already defined, checks that it is defined for the package
designated by the \texttt{actual-package}.
\item Exceptional situations
\label{sec:orgc29dab7}
An error of type \emph{package-error} is signaled when a package designated by the
\texttt{actual-package} or the \texttt{designated-package} does not exist.

If the \texttt{nickname} is one of the names \texttt{\#:CL}, \texttt{\#:COMMON-LISP} or \texttt{\#:KEYWORD}, an
error of type \emph{package-error} is signaled.

If the \texttt{nickname} is already defined to be a \emph{local nickname} for a package
different from the \texttt{actual-package}, a \emph{correctable} error of type \emph{package-error}
is signaled.
\item Implementation dependent
\label{sec:orgca9cb38}
The consequences are undefined when the \texttt{designated-package} designates the
\texttt{\#:COMMON-LISP} package or the \texttt{\#:KEYWORD} package.

(\textbf{Contains proposals}: see \hyperref[sec:org1879459]{Issue 5}.)

If the \texttt{nickname} shadows the \emph{package name} or one of the \emph{global nicknames} of
the \texttt{designated-package}, a style warning might be issued.
\end{enumerate}
\subsubsection{remove-package-local-nickname}
\label{sec:org0d8dfe3}
\begin{verbatim}
(remove-package-local-nickname old-nickname &optional designated-package)
  => nickname-removed-p
\end{verbatim}
\begin{enumerate}
\item Arguments and Values
\label{sec:orgfa10da8}
\texttt{old-nickname} --- a \emph{string designator}.

\texttt{designated-package} --- a \emph{package designator}.
The default is the \emph{current package}.

\texttt{nickname-removed-p} --- \emph{generalized boolean}.
\item Description
\label{sec:org1981974}
If \texttt{old-nickname} is defined to be a \emph{local nickname} in the \texttt{designated-package},
it is removed.

[Also see \hyperref[sec:orgede3d6c]{Issue 1}.] Returns \emph{true} if it removes a nickname, and \texttt{NIL} otherwise.
\item Exceptional situations
\label{sec:org910bc06}
An error of type \emph{package-error} is signaled when a package designated by the
\texttt{designated-package} does not exist.
\end{enumerate}
\subsubsection{package-local-nicknames}
\label{sec:orgae64d98}
\begin{verbatim}
(package-local-nicknames package-designator)
  => local-nicknames-alist
local-nicknames-alist ::= ((nickname . package)*)
\end{verbatim}
\begin{enumerate}
\item Arguments and Values
\label{sec:orge5a1026}
\texttt{package-designator} --- a \emph{package designator}.

\texttt{local-nicknames-alist} --- an \emph{alist}.

\texttt{nickname} --- a \emph{string}.

\texttt{package} --- a \emph{package}.
\item Description
\label{sec:org8d538a1}
Returns an \emph{alist} describing \emph{local nicknames} defined in the package designated
by the \texttt{package-designator}.
\item Exceptional situations
\label{sec:orgae28f80}
An error of type \texttt{package-error} is signaled when a package designated by the
\texttt{package-designator} does not exist.
\item Notes
\label{sec:orga321966}
The returned \emph{alist} must be safe to be modified by the user.
\end{enumerate}
\subsubsection{package-locally-nicknamed-by-list}
\label{sec:orgecd6c2d}
\begin{verbatim}
(package-locally-nicknamed-by-list package-designator)
  => packages-list
\end{verbatim}
\begin{enumerate}
\item Arguments and Values
\label{sec:orgc563c31}
\texttt{package-designator} --- a \emph{package designator}.

\texttt{packages-list} --- a \emph{list} of \emph{package} objects.
\item Description
\label{sec:orgd97f977}
Returns a \emph{list} of packages that have a \emph{local nickname} defined for the package
designated by the \texttt{package-designator}.
\item Exceptional situations
\label{sec:org21f85e7}
An error of type \texttt{package-error} is signaled when a package designated by the
\texttt{package-designator} does not exist.
\item Notes
\label{sec:org56d5887}
The returned \emph{list} must be safe to be modified by the user.
\end{enumerate}
\subsection{Affected symbols}
\label{sec:orgada03f2}
\subsubsection{defpackage}
\label{sec:org3495ff5}
See \hyperref[sec:orgbd9e3bf]{defpackage}.
\subsubsection{make-package}
\label{sec:org40d2ffe}
See \hyperref[sec:org1fd55bd]{make-package}.
\subsubsection{find-package}
\label{sec:org5d45ce9}
(\textbf{Contains proposals}: see \hyperref[sec:org8b64547]{Issue 3}, \hyperref[sec:orgbc62926]{Issue 8}.)

When the argument to \texttt{find-package} is a \emph{local nickname} defined in the \emph{current
package}, it returns the package nicknamed by this nickname.

This also affects all implied calls to \texttt{find-package}, including but not limited
to those performed by the lisp reader as well as those performed by \texttt{defpackage},
\texttt{make-package}, \texttt{export}, \texttt{find-symbol}, \texttt{import}, \texttt{rename-package}, \texttt{shadow},
\texttt{shadowing-import}, \texttt{delete-package}, \texttt{with-package-iterator}, \texttt{unexport},
\texttt{unintern}, \texttt{in-package}, \texttt{unuse-package}, \texttt{use-package}, \texttt{do-symbols},
\texttt{do-external-symbols}, \texttt{do-all-symbols}, \texttt{intern}, \texttt{package-name},
\texttt{package-nicknames}, \texttt{package-shadowing-symbols}, \texttt{package-use-list},
\texttt{package-used-by-list}.

\texttt{add-package-local-nickname}, \texttt{remove-package-local-nickname},
\texttt{package-local-nicknames} and \texttt{package-locally-nicknamed-by} are also affected.

The only exception is the \emph{tilde slash} directive of \texttt{format}, which should \textbf{not}
use \emph{local nicknames} from any package when looking up the specified symbol.
\subsubsection{rename-package}
\label{sec:orgb58c49b}
When a package is renamed with \texttt{rename-package}, it retains all \emph{local nicknames}
it has defined, as well as all \emph{local nicknames} by which it is nicknamed.

\begin{enumerate}
\item Implementation dependent
\label{sec:orgd44bf03}
(\textbf{Contains proposals}: see \hyperref[sec:org1879459]{Issue 5}.)

If the \emph{new-name} or one of the \emph{new-nicknames} is shadowed by one of the \emph{local
nicknames} of the package being renamed, a style warning might be issued.
\end{enumerate}
\subsubsection{delete-package}
\label{sec:org1108358}
When a package is deleted with \texttt{delete-package}, all \emph{local nicknames} defined in
that package are removed, as well as all \emph{local nicknames} by which it is
nicknamed.

This also means that a deleted package must not be available via calls to
\texttt{package-locally-nicknamed-by-list} and \texttt{package-local-nicknames}.
\subsubsection{format}
\label{sec:orgb164efb}
See \hyperref[sec:orgbc62926]{Issue 8}.
\subsubsection{$\backslash$*features$\backslash$*}
\label{sec:org8bf01d2}
If an implementation supports package-local nicknames, it should add symbols
\texttt{:package-local-nicknames} and \texttt{:cdr-NN} (per CDR 14) to \texttt{*features*}.
\subsection{Examples}
\label{sec:org6a49a9c}
[TODO]
\section{ISSUES}
\label{sec:org298932c}

\subsection{Issue 1 ([ADD-/REMOVE-]PACKAGE-LOCAL-NICKNAME return values)}
\label{sec:orgede3d6c}
\subsubsection{Description}
\label{sec:org3c2b081}
Return values of \texttt{add-package-local-nickname} and \texttt{remove-package-local-nickname}
are inconsistent. The first one always returns \emph{designated package}, while the
second one returns \emph{true} (generalized boolean) if a nickname was removed.

Moreover, there is no consensus among current implementations as to what the second
function (\texttt{remove-package-local-nickname}) should return.
\subsubsection{Examples}
\label{sec:orgef28080}
\begin{verbatim}
(defpackage #:foo (:use))
(defpackage #:bar (:use))

(add-package-local-nickname '#:nick '#:bar '#:foo)
; => #<PACKAGE "FOO">  (sbcl, ccl, ecl, acl, abcl, clasp, lispworks)
(add-package-local-nickname '#:nick '#:bar '#:foo)
; => #<PACKAGE "FOO">  (sbcl, ccl, ecl, acl, abcl, clasp, lispworks)
(remove-package-local-nickname '#:nick '#:foo)
; => T  (sbcl, ccl, ecl, acl, clasp)
; => #<PACKAGE "BAR">  (abcl)
; => NIL  (lispworks)
(remove-package-local-nickname '#:nick '#:foo)
; => NIL  (sbcl, ccl, ecl, acl, abcl, clasp, lispworks)
\end{verbatim}
\subsubsection{Current behavior}
\label{sec:orga96fc07}
sbcl, ccl, ecl, acl, abcl, clasp, lispworks:
  \texttt{add-package-local-nickname} always returns \emph{designated package}.

sbcl, ccl, ecl, acl, clasp:
  \texttt{remove-package-local-nickname} returns \texttt{T} if a nickname was removed,
  and \texttt{NIL} otherwise.

abcl:
  \texttt{remove-package-local-nickname} if a nickname was removed, returns \emph{true} (more
  specifically - the package that was nicknamed by that nickname), and \texttt{NIL}
  otherwise.

lispworks:
  \texttt{remove-package-local-nickname} always returns \texttt{NIL}.
\subsubsection{Proposal DESIGNATED-PACKAGE-IF-SUCCESSFUL}
\label{sec:orgdfa9235}
\begin{itemize}
\item \texttt{add-package-local-nickname} should return \emph{designated package} if a new nickname
was added and \texttt{NIL} otherwise, if the nickname already existed.
\item \texttt{remove-package-local-nickname} should return \emph{designated package} if a nickname
was removed and \texttt{NIL} otherwise.
\end{itemize}
\subsubsection{Proposal ALWAYS-T}
\label{sec:org5f8e5b6}
Similar to \texttt{use-package} and \texttt{unuse-package}:
\begin{itemize}
\item \texttt{add-package-local-nickname} should always return \texttt{T}.
\item \texttt{remove-package-local-nickname} should always return \texttt{T}.
\end{itemize}
\subsubsection{Proposal ELIMINATE-GENERALIZED-BOOLEAN}
\label{sec:org667c7e9}
\begin{itemize}
\item \texttt{add-package-local-nickname} should always return \emph{designated package}.
\item \texttt{remove-package-local-nickname} should return \texttt{T} if a nickname was removed and
\texttt{NIL} otherwise.
\end{itemize}

\subsection{Issue 2 (PRINT-READ consistency)}
\label{sec:orgf15eb69}
\subsubsection{Description}
\label{sec:orgc4ee2d8}
Lisp reader uses \texttt{find-package} when reading a symbol, which is affected by the
\emph{local nicknames} of the \emph{current package}. That means that to maintain \textbf{print-read}
consistency when printing a symbol, a good \emph{package prefix} must be used - such that
calling \texttt{find-package} on it in the \emph{current package} returns the symbol's \emph{home
package}.

There are several situations to consider:
\begin{enumerate}
\item Symbol is \emph{apparently uninterned}.

\emph{In this case it must be printed without any package prefix, preceeded by \texttt{\#:}.}

\item Symbol is accessible in the \emph{current package}.

\emph{In this case it must be printed without any package prefix.}

\item Symbol's \emph{home package} \emph{name} or one of its \emph{global nicknames} is not shadowed
by any \emph{local nickname} defined in the \emph{current package}.

\emph{In this case that name or global nickname can be used as the package prefix.}

\item There \textbf{exists} a \emph{local nickname} defined in the \emph{current package} for the
symbol's \emph{home package}.

\emph{In this case that local nickname can be used as the package prefix.}

\item Symbol's \emph{home package} \emph{name} and all of its \emph{global nicknames} are shadowed by
the \emph{local nicknames} of the \emph{current package} and there \textbf{is no} \emph{local nickname}
defined in the \emph{current package} for the symbol's \emph{home package}.

\emph{It is not clear how the symbol must be printed, see PROPOSALS.}
\end{enumerate}
\subsubsection{Examples}
\label{sec:orgbda42af}
\begin{verbatim}
(defpackage #:foo
  (:use)
  (:export #:+))

(defpackage #:bar
  (:use #:cl)
  (:local-nicknames (#:foo #:cl)))

(let ((*package* (find-package '#:bar)))
  (print 'foo:+))
; >> FOO:+  (sbcl, ccl, ecl, acl, abcl, clasp, lispworks)

;; In the package #:BAR symbol FOO:+ refers to CL:+
\end{verbatim}

\begin{verbatim}
(defpackage #:foo-a (:use) (:export #:quux))
(defpackage #:foo-b (:use) (:export #:quux))

(defpackage #:bar
  (:use)
  (:local-nicknames (#:foo-a #:foo-b)
                    (#:foo-b #:foo-a)))

(let ((*package* (find-package '#:bar)))
  (print 'foo-a:quux))
; >> FOO-B:QUUX  (sbcl, ccl, abcl, lispworks)
; >> FOO-A:QUUX  (ecl, acl, clasp)

;; In the package #:BAR symbol FOO-A:QUUX refers to FOO-B:QUUX
\end{verbatim}
\subsubsection{Current behavior}
\label{sec:org89a127b}
sbcl, ccl, abcl, lispworks:
  When exists in the \emph{current package}, a \emph{local nickname} is used as a package
  prefix when printing a symbol.

ecl, acl, clasp:
  \emph{local nickname} is never used as a package prefix when printing a symbol.
\subsubsection{Proposal SHARPSIGN-DOT}
\label{sec:orgdaa0269}
In the case (5) the symbol must be printed using the \texttt{\#.} syntax:

\begin{verbatim}
#.(cl:let ((cl:*package* (cl:find-package "KEYWORD")))
    (cl:find-symbol "BAR" "FOO"))
;; or
#.(cl:let ((cl:*package* (cl:find-package "KEYWORD")))
    (cl:intern "BAR" "FOO"))
\end{verbatim}

Note that \texttt{\#:KEYWORD} name is reserved for the \texttt{\#:KEYWORD} package and
cannot be used as a \emph{local nickname} thus this expression will always
evaluate to the symbol \texttt{foo::bar}.

If \texttt{*read-eval*} is \emph{false} and \texttt{*print-readably*} is \emph{true} an error of type
\texttt{print-not-readable} must be signalled.
\subsubsection{Proposal SHARPSIGN-COLON}
\label{sec:orgf384602}
In case (5) the symbol should be printed using the \emph{extended \texttt{\#:} syntax}:
\begin{verbatim}
#:(package name)
#::(package name)
\end{verbatim}

\emph{Shinmera's idea}.
\subsubsection{Proposal SHARPSIGN-BACKQUOTE}
\label{sec:org0f0e882}
In case (5) the symbol must be printed using the new \texttt{\#`} syntax for reading an
expression ignoring \emph{local nicknames} in the \emph{current package}:
\begin{verbatim}
#`foo:bar
#`foo::bar
\end{verbatim}

It can be implemented roughly as follows:
\begin{verbatim}
(defun |#`-reader| (stream subchar arg)
  (declare (ignore subchar arg))
  (let* ((current-package *package*)
         (local-nicknames (package-local-nicknames current-package)))
    (loop for (nick . package) in local-nicknames
          do (remove-package-local-nickname nick current-package))
    (unwind-protect
      (read stream t nil t)
      (loop for (nick . package) in local-nicknames
            do (add-package-local-nickname nick package current-package)))))

(set-dispatch-macro-character #\# #\` #'|#`-reader|)
\end{verbatim}
It is \emph{implementation-dependent} whether \emph{local nicknames} are actually removed
from the \emph{current package} or not.
\subsubsection{Proposal PRINT-UNREADABLY}
\label{sec:org77f921a}
In the case (5) the symbol must be printed unreadably using the \texttt{\#<} syntax:
\begin{verbatim}
#<SYMBOL IN THE SHADOWED PACKAGE FOO:BAR>
#<SYMBOL IN THE SHADOWED PACKAGE FOO::BAR>
\end{verbatim}

(Specifics are \emph{implementation-dependent}.)

If \texttt{*print-readably*} is \emph{true}, an error of type \texttt{print-not-readable} must be
signalled.
\subsubsection{Proposal THREE-FOUR-PACKAGE-MARKERS}
\label{sec:org3dd89d0}
In the case (5) the symbol must be printed using \texttt{:::} and \texttt{::::} syntax as follows:
\begin{verbatim}
foo:::bar   ; same as (cl:find-symbol "BAR" "FOO") in the #:KEYWORD package
foo::::bar  ; same as (cl:intern "BAR" "FOO") in #:KEYWORD package
\end{verbatim}
\subsubsection{Links}
\label{sec:orga053a82}
See \href{https://www.lispworks.com/documentation/HyperSpec/Body/22\_acca.htm}{CLHS 22.1.3.3.1 Package Prefixes for Symbols}.

\subsection{Issue 3 (Local nicknames effect on DEFPACKAGE, MAKE-PACKAGE and others)}
\label{sec:org8b64547}
\subsubsection{Description}
\label{sec:orgf668ca7}
It is not clear whether \emph{local nicknames} of the \emph{current package} should affect
the resolution of package designators provided in \texttt{make-package} and \texttt{defpackage},
as well as other functions and macros taking a package designator as an argument
(\texttt{in-package}, \texttt{add-package-local-nickname} and others).
\subsubsection{Examples}
\label{sec:org9f13733}
\begin{verbatim}
(defpackage #:foo-a (:use) (:export #:x))
(defpackage #:foo-b (:use) (:export #:x))

(defpackage #:bar
  (:use #:cl)
  (:local-nicknames (#:foo-a #:foo-b)
                    (#:foo-b #:foo-a)))

(in-package #:bar)

(defpackage #:quux-1
  (:use #:foo-a))
(package-name (symbol-package 'quux-1::x))
; => "FOO-B"  (sbcl, ccl, acl, abcl, lispworks)
; => "FOO-A"  (ecl, clasp)

(make-package '#:quux-2 :use '(#:foo-a))
(package-name (symbol-package 'quux-2::x))
; => "FOO-B"  (sbcl, ccl, ecl, acl, abcl, clasp, lispworks)

(defpackage #:quux-3
  (:use)
  (:local-nicknames (#:foo #:foo-a)))
(let ((*package* (find-package '#:quux-3)))
  (package-name (find-package '#:foo)))
; => "FOO-B"  (ccl, ecl, acl, abcl)
; => "FOO-A"  (sbcl, clasp, lispworks)

(import (car (find-all-symbols (string '#:add-package-local-nickname))))
(defpackage #:quux-4
  (:use))
(add-package-local-nickname '#:foo '#:foo-a '#:quux-4)
(let ((*package* (find-package '#:quux-4)))
  (package-name (find-package '#:foo)))
; => "FOO-B"  (ccl, ecl, clasp, abcl, lispworks)
; => "FOO-A"  (sbcl)

(use-package '#:foo-a '#:quux-4)
(package-name (symbol-package 'quux-4::x))
; => "FOO-B"  (sbcl, ccl, ecl, clasp, abcl, lispworks)
\end{verbatim}
\subsubsection{Current behavior}
\label{sec:org6ad2be2}
sbcl, lispworks:
  only \texttt{:local-nicknames} clause is \textbf{not} affected by \emph{local nicknames}.

ccl, acl, abcl:
  all options are affected.

ecl:
  only the \texttt{:local-nicknames} clause and keyword arguments (\texttt{:use} and
  \texttt{:local-nicknames}) are affected.

clasp:
  only keyword argument \texttt{:use} is affected.

Known exceptions in other functions/macros:
sbcl: \texttt{add-package-local-nickname} is not affected.
lispworks: \texttt{in-package} is not affected.
\subsubsection{Proposal ALL-AFFECTED}
\label{sec:org5004257}
All \texttt{defpackage} clauses (\texttt{:use}, \texttt{:local-nicknames}, \texttt{:import-from},
\texttt{:shadowing-import-from}) as well as all keyword arguments to \texttt{make-package}
(\texttt{:use} and \texttt{:local-nicknames}) must be affected by the \emph{local nicknames} of the
\emph{current package}.

All functions and macros taking a package designator as an argument must be
affected as well.

A non-exhaustive list of affected functions and macros:
  \texttt{export}, \texttt{find-symbol}, \texttt{import}, \texttt{rename-package}, \texttt{shadow},
  \texttt{shadowing-import}, \texttt{delete-package}, \texttt{with-package-iterator}, \texttt{unexport},
  \texttt{unintern}, \texttt{in-package}, \texttt{unuse-package}, \texttt{use-package}, \texttt{do-symbols},
  \texttt{do-external-symbols}, \texttt{do-all-symbols}, \texttt{intern}, \texttt{package-name},
  \texttt{package-nicknames}, \texttt{package-shadowing-symbols}, \texttt{package-use-list},
  \texttt{package-used-by-list}, \texttt{add-package-local-nickname},
  \texttt{remove-package-local-nickname}, \texttt{package-local-nicknames},
  \texttt{package-locally-nicknamed-by}.

\subsection{Issue 4 (Local nicknames of the package being defined)}
\label{sec:orgd561e6a}
\subsubsection{Description}
\label{sec:org679de4f}
It is not clear whether \emph{local nicknames} of the package \textbf{being defined} should
affect \texttt{make-package} or \texttt{defpackage}.
\subsubsection{Examples}
\label{sec:orgf8201c0}
\begin{verbatim}
(defpackage #:foo-a (:use) (:export #:x))
(defpackage #:foo-b (:use) (:export #:x))

(defpackage #:bar
  (:local-nicknames (#:foo-a #:foo-b)
                    (#:foo-b #:foo-a))
  (:use #:foo-a))

(package-name (symbol-package 'bar::x))
; => "FOO-A"  (sbcl, ccl, acl, abcl, clasp, lispworks)
; => "FOO-B"  (ecl)
\end{verbatim}
\subsubsection{Current behavior}
\label{sec:org51ee048}
sbcl, ccl, acl, abcl, lispworks: not affected

ecl: \texttt{:use}, \texttt{:import-from} and \texttt{:shadowing-import-from} are affected.

clasp: \texttt{:local-nicknames} is affected by previous \texttt{:local-nicknames} clauses.
\subsubsection{Proposal NO-EFFECT}
\label{sec:orgf637172}
Local nicknames of a package being defined should not affect other defpackage
clauses (\texttt{:use}, \texttt{:local-nicknames}, \texttt{:import-from}, \texttt{:shadowing-import-from}).

The keyword argument \texttt{:local-nicknames} to \texttt{make-package} should not affect the
\texttt{:use} keyword argument either.

\subsection{Issue 5 (Local nickname shadowing package's own name)}
\label{sec:org1879459}
\subsubsection{Description}
\label{sec:org1c5ce30}
It is not clear whether it is valid to have a \emph{local nickname} in a package
shadowing its own name or nickname.
\subsubsection{Examples}
\label{sec:orgfcf218f}
\begin{verbatim}
(defpackage #:foo
  (:use)
  (:nicknames #:bar)
  (:local-nicknames (#:foo #:cl)
                    (#:bar #:cl)))
; => continuable error  (sbcl, ccl, abcl)
; => error  (lispworks)
; => ok  (ecl, acl, clasp)
\end{verbatim}
\subsubsection{Current behavior}
\label{sec:orgd2fd682}
sbcl, ccl, abcl and lispwork signal an error.
\subsubsection{Proposal ALLOW}
\label{sec:orgd7ee6f2}
It should be allowed to use package's own name or global nickname, but a
style-warning can be signalled.
\begin{enumerate}
\item Rationale
\label{sec:org49984eb}
Such local nicknames are not likely to break anything. Even though they can be a
bit confusing, this fact alone doesn't warrant an error being signalled.

Moreover, on all implementations it is possible to obtain such nicknames, even if
on some of them invoking the \texttt{continue} restart is going to be needed. This also
suggests that cost of adoption is very low.
\end{enumerate}

\subsection{Issue 6 (Additional keyword argument to MAKE-PACKAGE)}
\label{sec:orgd3fb702}
\subsubsection{Current behavior}
\label{sec:org04ba437}
sbcl, ccl, abcl, clasp, lispworks: no additional key argument.

ecl: has an additional keyword argument \texttt{:local-nicknames}, but it is undocumented
and it segfaults on incorrect usage. The expected value is a list of conses:
\texttt{((nickname . package)*)}.

acl: has an additional keyword argument \texttt{:local-nicknames}. The expected value is
a list of lists: \texttt{((nickname package)*)}.
\subsubsection{Proposal EXTRA-KEYWORD-ARGUMENT}
\label{sec:org09457bb}
Add \texttt{:local-nicknames} keyword argument to \texttt{make-package}:
\begin{verbatim}
local-nicknames ::= ((nickname package)*)
\end{verbatim}

\texttt{nickname} must be a \emph{string designator}.
\texttt{package} must be a \emph{package designator}.

\texttt{local-nicknames} defaults to an \emph{empty list}.

See \hyperref[sec:org1fd55bd]{make-package}.

\subsection{Issue 7 (Multiple local nicknames)}
\label{sec:org33fe300}
\subsubsection{Description}
\label{sec:org8a6ba3f}
It is not clear whether \texttt{package-locally-nicknamed-by-list} should be allowed to return
lists with duplicate entries (when there are multiple local nicknames in one package).
\subsubsection{Examples}
\label{sec:org7016090}
\begin{verbatim}
(defpackage #:foo
  (:use)
  (:local-nicknames (#:bar #:cl)
                    (#:baz #:cl)))
(mapcar #'package-name (package-locally-nicknamed-by-list '#:cl))
; => ("FOO")  (sbcl, acl, clasp, lispworks)
; => ("FOO" "FOO")  (ccl, ecl, abcl)
\end{verbatim}
\subsubsection{Current behavior}
\label{sec:org0d0ba72}
sbcl, acl, clasp, lispworks:
  \texttt{package-locally-nicknamed-by-list} never contains duplicate entries.

ccl, abcl, ecl:
  \texttt{package-locally-nicknamed-by-list} might contain duplicate entries.
\subsubsection{Proposal NO-DUPLICATES}
\label{sec:org38cc840}
\texttt{package-locally-nicknamed-by-list} must return a list without duplicate entries.

\subsection{Issue 8 (Interaction with FORMAT)}
\label{sec:orgbc62926}
\emph{by |3b|}
\subsubsection{Description}
\label{sec:org528717b}
It is not clear how \emph{local nicknames} should affect \texttt{format}'s \texttt{\textbackslash{}\textasciitilde{}//} directive.

First, it would be inconvinient if it wouldn't be possible to use \emph{local
nicknames} with the \texttt{\textbackslash{}\textasciitilde{}//} directive.

Secondly, it would be unintuitive if the function used would depend on the
\emph{current package} at the \textbf{execution} time. This also might break [existing] code,
if a local nickname in the \emph{current package} shadows a package that contains a
function used by a control string in another function.

Finally, if the call to \texttt{format} is not compiled, it is hard to impossible to find
the function using \emph{local nicknames} of the package that was the \emph{current package}
at the \textbf{compile} time.
\subsubsection{Examples}
\label{sec:orgdcef506}
\begin{verbatim}
(defpackage #:foo-a (:use) (:export #:ff))
(defpackage #:foo-b (:use) (:export #:ff))

(defun foo-a:ff (stream &rest args)
  (declare (ignore args))
  (format stream "FOO-A:FF"))

(defun foo-b:ff (stream &rest args)
  (declare (ignore args))
  (format stream "FOO-B:FF"))

(defpackage #:bar-a
  (:use #:cl)
  (:local-nicknames (#:nick #:foo-a)))

(defpackage #:bar-b
  (:use #:cl)
  (:local-nicknames (#:nick #:foo-b)))

(in-package #:bar-a)

(defun test ()
  (format t "Called ~/nick:ff/ & " nil)
  (let ((*package* (find-package (quote #:bar-a))))  ; or #.*package*
    (format t "~/nick:ff/~%" nil)))

(test)
; => "Called FOO-A:FF & FOO-A:FF"  (sbcl, ccl, ecl, acl, abcl, clasp)
; lispworks errors (NICK package not found)
(let ((*package* (find-package (quote #:bar-b))))
  (test))
; => "Called FOO-A:FF & FOO-A:FF"  (sbcl, clasp)
; => "Called FOO-B:FF & FOO-A:FF"  (ccl, ecl, acl, abcl)
; lispworks errors (NICK package not found)
\end{verbatim}
\subsubsection{Proposal NO-LOCAL-NICKNAMES}
\label{sec:orgcb58369}
\texttt{format}'s \texttt{\textbackslash{}\textasciitilde{}//} directive must \textbf{not} use \emph{local nicknames} of any package when
looking up the specified symbol.

Rationale: In the spirit of how when the package is not specified, the symbol is
not looked up in the \emph{current package}, but instead in the \texttt{\#:CL-USER} package;
the \emph{tilde slash} directive should not depend on the value of \texttt{*package*} at any
time.  Specifying it to use \emph{local nicknames} of the \texttt{\#:CL-USER} package instead
would risk breaking the existing code when adding local nicknames to that package.
\subsubsection{Links}
\label{sec:org23cf66d}
See \href{https://www.lispworks.com/documentation/HyperSpec/Body/22\_ced.htm}{CLHS 22.3.5.4 Tilde Slash: Call Function}.

\subsection{Issue 9 (Empty package local name)}
\label{sec:org96a0ed7}
\subsubsection{Description}
\label{sec:org93be394}
It is not clear whether it should be allowed to use \texttt{""} as a local nickname,
and in the case this is allowed, whether it should affect the \texttt{:xxxx} syntax.
\subsubsection{Examples}
\label{sec:org29729b4}
\begin{verbatim}
(defpackage #:foo
  (:use #:cl)
  (:local-nicknames ("" #:cl)))

(in-package #:foo)

(package-name (symbol-package ':*package*))
; => "KEYWORD"  (sbcl, ccl, ecl, abcl, clasp, lispworks)
; => "COMMON-LISP"  (acl)

(package-name (symbol-package '||:*package*))
; => "KEYWORD"  (ecl, clasp, lispworks)
; => "COMMON-LISP"  (sbcl, ccl, acl)
; abcl errors
\end{verbatim}
\subsubsection{Current behavior}
\label{sec:org3e9b33b}
sbcl, ccl:
\texttt{:xxxx} is read as a keyword;
\texttt{||:xxxx} is read as a symbol in the package named or nicknamed \texttt{""}.

ecl, clasp, lispworks:
\texttt{||:xxxx} and \texttt{:xxxx} are read as a keyword.

acl:
\texttt{:xxxx} and \texttt{||:xxxx} are read as a symbol in the package named or nicknamed \texttt{""}.
\texttt{""} is by default a global nickname for the \texttt{\#:KEYWORD} package.

abcl:
\texttt{:xxxx} is read as a keyword;
\texttt{||:xxxx} syntax cannot be read (attemts result in an error).
\subsubsection{Proposal ALLOW-BUT-KEEP-KEYWORDS}
\label{sec:orgfa2a433}
The \texttt{""} local nickname should be explicitely allowed. \texttt{:xxxx} should be always
read as a keyword regardless of package names or nicknames. \texttt{||:xxxx} should be
read as a symbol in the package named or nicknamed by \texttt{""}.
\subsubsection{Proposal ALLOW-FUN}
\label{sec:orgb0d7265}
The \texttt{""} local nickname should be explicitely allowed. Both \texttt{:xxxx} and \texttt{||:xxxx}
should be read as a symbol in the package named or nicknamed by \texttt{""}.
\subsubsection{Links}
\label{sec:orgab1a08e}
See \href{https://github.com/s-expressionists/wscl/issues/63}{WSCL issue 63}.
\section{Links}
\label{sec:orge268f2b}
3b's \href{https://github.com/3b/package-local-nicknames/blob/master/docs.org}{notes} on package-local nicknames.

phoe's \href{https://github.com/phoe/trivial-package-local-nicknames}{tests}.

SBCL's \href{https://www.sbcl.org/manual/\#Package\_002dLocal-Nicknames}{manual entry}.

Section 4.3 of the ABCL's manual. (\href{https://github.com/armedbear/abcl/blob/master/doc/manual/abcl.tex\#L1249}{\TeX{} file on github})
\section{Copying and License}
\label{sec:orgdf25b15}
[TODO]
\end{document}