% Created 2024-07-07 Sun 13:24
% Intended LaTeX compiler: pdflatex
\documentclass[11pt]{article}
\usepackage[utf8]{inputenc}
\usepackage[T1]{fontenc}
\usepackage{graphicx}
\usepackage{longtable}
\usepackage{wrapfig}
\usepackage{rotating}
\usepackage[normalem]{ulem}
\usepackage{amsmath}
\usepackage{amssymb}
\usepackage{capt-of}
\usepackage{hyperref}
\usepackage[margin=1in]{geometry}
\author{Gleefre}
\date{\today}
\title{Package-Local Nicknames}
\hypersetup{
 pdfauthor={Gleefre},
 pdftitle={Package-Local Nicknames},
 pdfkeywords={},
 pdfsubject={This is a CDR specification for package-local nicknames.},
 pdfcreator={Emacs 28.2 (Org mode 9.5.5)}, 
 pdflang={English}}
\begin{document}

\maketitle
\tableofcontents

[THIS IS A DRAFT]

\section{Specification}
\label{sec:org9130466}
\subsection{Introduction}
\label{sec:org7c66c90}
This is a specification for the \texttt{Package-Local Nicknames} extension in Common Lisp.
\subsubsection{Rationale}
\label{sec:org4328afb}
Package-local nicknames make it possible to use short and easy-to-use names
without potentially introducing name conflicts as can happen with usual nicknames.
\subsubsection{Current state}
\label{sec:org83085f3}
Package-local nicknames are implemented in some form in \texttt{SBCL}, \texttt{CCL}, \texttt{ECL},
\texttt{Clasp}, \texttt{ABCL}, \texttt{Allegro CL}, \texttt{LispWorks}. There is also a pending MR for the
\texttt{CLISP} implementation.

Unfortunately, there are multiple inconsistencies between implementations. All of
them lose \emph{print-read consistency} to some extent, and there are multiple edge
cases that aren't always implemented correctly or in the same way.
\subsubsection{Goal}
\label{sec:org713b498}
The purpose of this document is to standardize the \texttt{Package-Local Nicknames}
extension and to address some existing issues.

[TODO] This CDR also aims to provide an extensive test suite for the extension.
\subsection{Description}
\label{sec:orgd73d9fe}
A \emph{package-local nickname} (or a \emph{local nickname}) defined in some \emph{designated
package} has the same effects as a usual \emph{package nickname} (later referred to as a
\emph{global nickname}), except that these effects only apply when \texttt{*package*} is bound
to that \emph{designated package}.

This means that a call to \texttt{find-package} with a \emph{local nickname} that is defined in
the \emph{current package} returns the package nicknamed by this nickname. This also
affects all implied calls to \texttt{find-package}, including those performed by the Lisp
reader.

In addition, to maintain \emph{print-read consistency}, the Lisp printer is affected by
\emph{local nicknames} defined in the \emph{current package}.  For details see \hyperref[sec:org260b361]{Issue 2}.

A \emph{local nickname} is allowed to shadow a \emph{package name} or a \emph{global nickname},
except for the names \texttt{\#:CL}, \texttt{\#:COMMON-LISP} and \texttt{\#:KEYWORD} which must always
refer to their packages. The consequences of adding \emph{local nicknames} to the
packages \texttt{\#:COMMON-LISP} and \texttt{\#:KEYWORD} are also undefined.
\subsection{API}
\label{sec:orgaecdb27}
\subsubsection{defpackage}
\label{sec:orge74469a}
\begin{enumerate}
\item Description
\label{sec:orga8e8291}
The \texttt{defpackage} options are extended to include the \emph{local-nicknames-option}:
\begin{verbatim}
local-nicknames-option ::= (:local-nicknames (nickname package)*)
\end{verbatim}


Each pair specifies a \emph{local nickname} \texttt{nickname} for the corresponding \texttt{package}.

This option may appear more than once.
\item Arguments and Values:
\label{sec:orgd02efe8}
\texttt{nickname} --- a \emph{string designator}.

\texttt{package} --- a \emph{package designator}.
\item Exceptional situations
\label{sec:org2a09caf}
An error of type \emph{package-error} is signaled when a package designated by the
\texttt{package} does not exist.

Name conflict errors are handled by the underlying calls to
\texttt{make-package} and \texttt{add-package-local-nickname}.
\item Implementation dependent
\label{sec:org0ab63ba}
The consequences are undefined when a \emph{local nickname} is specified for the
package that is being defined. (See \hyperref[sec:org43f3224]{Issue 4}.)

The consequences are undefined when supplied \emph{local nicknames} are at variance
with the current state of the package. An implementation might choose to remove
all existing \emph{local nicknames} at the beginning of each redefinition of the
package.
\end{enumerate}
\subsubsection{make-package}
\label{sec:org11654fe}
\begin{enumerate}
\item Description
\label{sec:orge529f42}
(\textbf{Contains proposals}: see \hyperref[sec:org09ee005]{Issue 6}.)

The \texttt{make-package} lambda list is extended to include an additional keyword
argument \texttt{:local-nicknames}:
\begin{verbatim}
local-nicknames ::= ((nickname package)*)
\end{verbatim}


\texttt{local-nicknames} specifies zero or more \emph{local nicknames} to be defined in the
new \emph{package}.
\item Arguments and Values:
\label{sec:org9bc8ff7}
\texttt{local-nicknames} --- a \emph{list} of pairs of form \texttt{(nickname package)}.
The default is an \emph{empty list}.

\texttt{nickname} --- a \emph{string designator}.

\texttt{package} --- a \emph{package designator}.
\item Exceptional situations
\label{sec:org31b71c6}
An error of type \emph{package-error} is signaled when a package designated by the
\texttt{package} does not exist.

If the \texttt{nickname} is one of the names \texttt{\#:CL}, \texttt{\#:COMMON-LISP} or \texttt{\#:KEYWORD}, an
error of type \emph{package-error} is signaled.

If two or more local nicknames result in a name conflict, a \emph{correctable} error
of type \emph{package-error} is signaled. A name conflict occurs when multiple local
nicknames have same nicknames (equal by \texttt{string=}) but different packages.

A name conflict between multiple local nicknames may be resolved in favor of
either nickname being defined.
\item Implementation dependent
\label{sec:orgc594fde}
The consequences are undefined when a \emph{local nickname} is specified for the
package that is being defined. (See \hyperref[sec:org43f3224]{Issue 4}.)
\end{enumerate}
\subsubsection{add-package-local-nickname}
\label{sec:orgd3afceb}
\begin{verbatim}
(add-package-local-nickname nickname actual-package &optional designated-package)
  => designated-package-object
\end{verbatim}
\begin{enumerate}
\item Arguments and Values
\label{sec:org980d595}
\texttt{nickname} --- a \emph{string designator}.

\texttt{actual-package} --- a \emph{package designator}.

\texttt{designated-package} --- a \emph{package designator}.
The default is the \emph{current package}.

\texttt{designated-package-object} --- a \emph{package}.
\item Description
\label{sec:orgb108cb4}
Defines a \emph{package-local nickname} \texttt{nickname} for the \texttt{actual-package} in the
\texttt{designated-package}.

[Also see \hyperref[sec:org0a8d60e]{Issue 1}.] Returns the package designated by the \texttt{designated-package}.

If the \emph{nickname} is already defined, checks that it is defined for the package
designated by the \texttt{actual-package}. If a name conflict occurs, restarts \texttt{abort}
and \texttt{continue} can be used to correct the error.

If the \texttt{continue} restart is invoked, the existing \emph{local nickname} is removed
and the new nickname is defined.

If the \texttt{abort} restart is invoked, the existing nickname is not removed, and the
new nickname is not defined.
\item Exceptional situations
\label{sec:org7a6d71d}
An error of type \emph{package-error} is signaled when a package designated by the
\texttt{actual-package} or the \texttt{designated-package} does not exist.

If the \texttt{nickname} is one of the names \texttt{\#:CL}, \texttt{\#:COMMON-LISP} or \texttt{\#:KEYWORD}, an
error of type \emph{package-error} is signaled.

If the \texttt{nickname} is already defined to be a \emph{local nickname} for another package
different from the \texttt{actual-package}, a \emph{correctable} error of type \emph{package-error}
is signaled.
\item Implementation dependent
\label{sec:org9e1f4dd}
The consequences are undefined when the \texttt{designated-package} designates the
\texttt{\#:COMMON-LISP} package or the \texttt{\#:KEYWORD} package.

(\textbf{Contains proposals}: see \hyperref[sec:org72887a2]{Issue 5}.)

If the \texttt{nickname} shadows the \emph{package name} or one of the \emph{global nicknames} of
the \texttt{designated-package}, a style warning might be issued.
\end{enumerate}
\subsubsection{remove-package-local-nickname}
\label{sec:orgc4ecab2}
\begin{verbatim}
(remove-package-local-nickname old-nickname &optional designated-package)
  => nickname-removed-p
\end{verbatim}
\begin{enumerate}
\item Arguments and Values
\label{sec:org353616e}
\texttt{old-nickname} --- a \emph{string designator}.

\texttt{designated-package} --- a \emph{package designator}.
The default is the \emph{current package}.

\texttt{nickname-removed-p} --- \emph{generalized boolean}.
\item Description
\label{sec:org2577866}
If \texttt{old-nickname} is defined to be a \emph{local nickname} in the \texttt{designated-package},
it is removed.

[Also see \hyperref[sec:org0a8d60e]{Issue 1}.] Returns \emph{true} if it removes a nickname, and \texttt{NIL} otherwise.
\item Exceptional situations
\label{sec:org43f5030}
An error of type \emph{package-error} is signaled when a package designated by the
\texttt{designated-package} does not exist.
\end{enumerate}
\subsubsection{package-local-nicknames}
\label{sec:org88c957d}
\begin{verbatim}
(package-local-nicknames package-designator)
  => local-nicknames-alist
local-nicknames-alist ::= ((nickname . package)*)
\end{verbatim}
\begin{enumerate}
\item Arguments and Values
\label{sec:orgbe39964}
\texttt{package-designator} --- a \emph{package designator}.

\texttt{local-nicknames-alist} --- an \emph{alist}.

\texttt{nickname} --- a \emph{string}.

\texttt{package} --- a \emph{package}.
\item Description
\label{sec:org7069520}
Returns an \emph{alist} describing \emph{local nicknames} defined in the package designated
by the \texttt{package-designator}.
\item Exceptional situations
\label{sec:org740c9a7}
An error of type \emph{package-error} is signaled when a package designated by the
\texttt{package-designator} does not exist.
\item Notes
\label{sec:org3405fb0}
The returned \emph{alist} must be safe to be modified by the user.
\end{enumerate}
\subsubsection{package-locally-nicknamed-by-list}
\label{sec:org65f551f}
\begin{verbatim}
(package-locally-nicknamed-by-list package-designator)
  => packages-list
\end{verbatim}
\begin{enumerate}
\item Arguments and Values
\label{sec:org1978c9b}
\texttt{package-designator} --- a \emph{package designator}.

\texttt{packages-list} --- a \emph{list} of \emph{package} objects.
\item Description
\label{sec:org0a3c738}
Returns a \emph{list} of packages that have a \emph{local nickname} defined for the package
designated by the \texttt{package-designator}.
\item Exceptional situations
\label{sec:orge04c89c}
An error of type \emph{package-error} is signaled when a package designated by the
\texttt{package-designator} does not exist.
\item Notes
\label{sec:orge0a0b75}
The returned \emph{list} must be safe to be modified by the user.
\end{enumerate}
\subsection{Affected symbols}
\label{sec:orge2d2d1e}
\subsubsection{defpackage}
\label{sec:org027955d}
See \hyperref[sec:orge74469a]{defpackage}.
\subsubsection{make-package}
\label{sec:org96a3bfb}
See \hyperref[sec:org11654fe]{make-package}.
\subsubsection{find-package}
\label{sec:org98b302b}
(\textbf{Contains proposals}: see \hyperref[sec:orga21bf4f]{Issue 3}, \hyperref[sec:orgd77e416]{Issue 8}.)

When the argument to \texttt{find-package} is a \emph{local nickname} defined in the \emph{current
package}, it returns the package nicknamed by this nickname.

This also affects all implied calls to \texttt{find-package}, including but not limited
to those performed by the lisp reader as well as those performed by \texttt{defpackage},
\texttt{make-package}, \texttt{export}, \texttt{find-symbol}, \texttt{import}, \texttt{rename-package}, \texttt{shadow},
\texttt{shadowing-import}, \texttt{delete-package}, \texttt{with-package-iterator}, \texttt{unexport},
\texttt{unintern}, \texttt{in-package}, \texttt{unuse-package}, \texttt{use-package}, \texttt{do-symbols},
\texttt{do-external-symbols}, \texttt{do-all-symbols}, \texttt{intern}, \texttt{package-name},
\texttt{package-nicknames}, \texttt{package-shadowing-symbols}, \texttt{package-use-list},
\texttt{package-used-by-list}.

\texttt{add-package-local-nickname}, \texttt{remove-package-local-nickname},
\texttt{package-local-nicknames} and \texttt{package-locally-nicknamed-by} are also affected.

The only exception is the \emph{tilde slash} directive of \texttt{format}, which should \textbf{not}
use \emph{local nicknames} from any package when looking up the specified symbol.
\subsubsection{rename-package}
\label{sec:org21bc41b}
When a package is renamed with \texttt{rename-package}, it retains all \emph{local nicknames}
it has defined, as well as all \emph{local nicknames} by which it is nicknamed.

\begin{enumerate}
\item Implementation dependent
\label{sec:org97adaff}
(\textbf{Contains proposals}: see \hyperref[sec:org72887a2]{Issue 5}.)

If the \emph{new-name} or one of the \emph{new-nicknames} is shadowed by one of the \emph{local
nicknames} of the package being renamed, a style warning might be issued.
\end{enumerate}
\subsubsection{delete-package}
\label{sec:org0c77f21}
When a package is deleted with \texttt{delete-package}, all \emph{local nicknames} defined in
that package are removed, as well as all \emph{local nicknames} by which it is
nicknamed.

This also means that a deleted package must not be available via calls to
\texttt{package-locally-nicknamed-by-list} and \texttt{package-local-nicknames}.
\subsubsection{format}
\label{sec:org6951869}
See \hyperref[sec:orgd77e416]{Issue 8}.
\subsubsection{$\backslash$*features$\backslash$*}
\label{sec:orgd1a490d}
If an implementation supports package-local nicknames, it should add symbols
\texttt{:package-local-nicknames} and \texttt{:cdr-NN} (per CDR 14) to \texttt{*features*}.
\subsection{Examples}
\label{sec:orgf81a7eb}
[TODO]
\section{ISSUES}
\label{sec:org35c93ef}
\subsection{Issue 1 (ADD-/REMOVE-PACKAGE-LOCAL-NICKNAME return values)}
\label{sec:org0a8d60e}
\subsubsection{Description}
\label{sec:org50a52e8}
Functions \texttt{add-package-local-nickname} and \texttt{remove-package-local-nickname} have
inconsistent return values.

The first one always returns the \emph{designated package}, while the second one
returns \emph{true} (a \emph{generalized boolean}) that indicates whether a nickname was
removed.

Furthermore, there is no consensus among existing implementations as to what this
\emph{generalized boolean} represents.

In comparison,the functions \texttt{use-package} and \texttt{unuse-package} always return \texttt{T},
and the functions \texttt{export} and \texttt{unexport} have the same behavior. There are also
functions \texttt{unintern} and \texttt{delete-package} which return a \emph{generalized boolean}
indicating if the operation changed something, but their counterparts \texttt{intern} and
\texttt{make-package}, unlike with local nicknames, create and return an \emph{object} (a
\emph{symbol} or a \emph{package}).
\subsubsection{Examples}
\label{sec:org474e436}
\begin{verbatim}
(defpackage #:foo (:use))
(defpackage #:bar (:use))

(add-package-local-nickname '#:nick '#:bar '#:foo)
; => #<PACKAGE "FOO">  (sbcl, ccl, ecl, acl, abcl, clasp, lispworks)
(add-package-local-nickname '#:nick '#:bar '#:foo)
; => #<PACKAGE "FOO">  (sbcl, ccl, ecl, acl, abcl, clasp, lispworks)
(remove-package-local-nickname '#:nick '#:foo)
; => T  (sbcl, ccl, ecl, acl, clasp)
; => #<PACKAGE "BAR">  (abcl)
; => NIL  (lispworks)
(remove-package-local-nickname '#:nick '#:foo)
; => NIL  (sbcl, ccl, ecl, acl, abcl, clasp, lispworks)
\end{verbatim}
\subsubsection{Current behavior}
\label{sec:org9974eeb}
sbcl, ccl, ecl, acl, abcl, clasp, lispworks:
  \texttt{add-package-local-nickname} always returns \emph{designated package}.

sbcl, ccl, ecl, acl, clasp:
  \texttt{remove-package-local-nickname} returns \texttt{T} if a nickname was removed,
  and \texttt{NIL} otherwise.

abcl:
  \texttt{remove-package-local-nickname} returns \emph{true} if a nickname was removed, (more
  specifically, it returns the package nicknamed by the removed local nickname),
  and \texttt{NIL} otherwise.

lispworks:
  \texttt{remove-package-local-nickname} always returns \texttt{NIL}.
\subsubsection{Proposal ALWAYS-T}
\label{sec:org66208c3}
\texttt{add-package-local-nickname} should always return \texttt{T}.

\texttt{remove-package-local-nickname} should always return \texttt{T}.
\subsubsection{Proposal ELIMINATE-GENERALIZED-BOOLEAN}
\label{sec:org7c29eeb}
\texttt{add-package-local-nickname} should return the \emph{designated package}.

\texttt{remove-package-local-nickname} should return \texttt{T} if a nickname was removed and
\texttt{NIL} otherwise.
\subsubsection{Proposal DESIGNATED-PACKAGE-IF-SUCCESSFUL}
\label{sec:org6ab78a5}
\texttt{add-package-local-nickname} should return the \emph{designated package} if a new
nickname was added, and \texttt{NIL} otherwise.

\texttt{remove-package-local-nickname} should return the \emph{designated package} if a
nickname was removed, and \texttt{NIL} otherwise.
\subsubsection{Proposal NICKNAMED-PACKAGE-IF-SUCCESSFUL}
\label{sec:org3d37cbf}
\texttt{add-package-local-nickname} should return the \emph{nicknamed package} if a new
nickname was added, and \texttt{NIL} otherwise.

\texttt{remove-package-local-nickname} should return the previously \emph{nicknamed package}
if a nickname was removed, and \texttt{NIL} otherwise.
\subsection{Issue 2 (PRINT-READ consistency)}
\label{sec:org260b361}
\subsubsection{Description}
\label{sec:org7413535}
The Lisp reader uses \texttt{find-package} when reading a symbol, which is affected by the
\emph{local nicknames} of the \emph{current package}. That means that to maintain \textbf{print-read}
consistency when printing a symbol, a good \emph{package prefix} must be used - such that
calling \texttt{find-package} on it in the \emph{current package} returns the \emph{home package} of
the symbol.

There are several situations to consider:
\begin{enumerate}
\item The symbol is \emph{apparently uninterned}.

\emph{In this case it is printed without the package prefix, using the uninterned
symbol syntax \texttt{\#:}.}

\item The symbol is accessible in the \emph{current package}.

\emph{In this case it is printed without any package prefix.}

\item The \emph{name} or one of the \emph{global nicknames} of the \emph{home package} of the symbol
is not shadowed by any \emph{local nickname} defined in the \emph{current package}.

\emph{In this case that name or global nickname might be used as the package prefix.}

\item There exists a \emph{local nickname} defined in the \emph{current package} for the \emph{home
package} of the symbol.

\emph{In this case that local nickname might be used as the package prefix.}

\item Both the \emph{name} and all \emph{global nicknames} of the \emph{home package} of the symbol
are shadowed by \emph{local nicknames} of the \emph{current package}, and there is no
\emph{local nickname} defined in the \emph{current package} for the \emph{home package}.

\emph{It is not clear how the symbol is printed, see PROPOSALS.}
\end{enumerate}
\subsubsection{Examples}
\label{sec:org66aa78e}
\begin{verbatim}
(defpackage #:foo
  (:use)
  (:export #:+))

(defpackage #:bar
  (:use #:cl)
  (:local-nicknames (#:foo #:cl)))

(let ((*package* (find-package '#:bar)))
  (print 'foo:+))
; >> FOO:+  (sbcl, ccl, ecl, acl, abcl, clasp, lispworks)

;; In the package #:BAR symbol FOO:+ refers to CL:+
\end{verbatim}

\begin{verbatim}
(defpackage #:foo-a (:use) (:export #:quux))
(defpackage #:foo-b (:use) (:export #:quux))

(defpackage #:bar
  (:use)
  (:local-nicknames (#:foo-a #:foo-b)
                    (#:foo-b #:foo-a)))

(let ((*package* (find-package '#:bar)))
  (print 'foo-a:quux))
; >> FOO-B:QUUX  (sbcl, ccl, abcl, lispworks)
; >> FOO-A:QUUX  (ecl, acl, clasp)

;; In the package #:BAR symbol FOO-A:QUUX refers to FOO-B:QUUX
\end{verbatim}
\subsubsection{Current behavior}
\label{sec:org7c1bb87}
sbcl, ccl, abcl, lispworks:
  When possible, a \emph{local nickname} is used as a package prefix.

ecl, acl, clasp:
  \emph{Local nicknames} are never used as a package prefix.
\subsubsection{Proposal SHARPSIGN-DOT}
\label{sec:org0a8eba5}
In the 5th case the symbol is printed using the \texttt{\#.} syntax:

\begin{verbatim}
#.(cl:let ((cl:*package* (cl:find-package "KEYWORD")))
    (cl:find-symbol "BAR" "FOO"))
;; or
#.(cl:let ((cl:*package* (cl:find-package "KEYWORD")))
    (cl:intern "BAR" "FOO"))
\end{verbatim}

If \texttt{*read-eval*} is \emph{false} and \texttt{*print-readably*} is \emph{true}, an error of type
\texttt{print-not-readable} is signaled.
\begin{enumerate}
\item Note
\label{sec:org03752ad}
Since \texttt{\#:KEYWORD} cannot be used as a \emph{local nickname}, and no \emph{local nicknames}
can be defined in the \texttt{\#:KEYWORD} package, this expression is guaranteed to
evaluate to the symbol in the correct package.
\end{enumerate}
\subsubsection{Proposal SHARPSIGN-COLON}
\label{sec:org87a3be1}
In the 5th case the symbol is printed using the \emph{extended \texttt{\#:} syntax}:
\begin{verbatim}
#:(package name)
#::(package name)
\end{verbatim}

\emph{Shinmera's idea}.
\subsubsection{Proposal SHARPSIGN-BACKQUOTE}
\label{sec:orge3bed68}
In the 5th case the symbol is printed using the new \texttt{\#`} syntax for reading an
expression ignoring \emph{local nicknames} in the \emph{current package}:
\begin{verbatim}
#`foo:bar
#`foo::bar
\end{verbatim}

It can be implemented roughly as follows:
\begin{verbatim}
(defun |#`-reader| (stream subchar arg)
  (declare (ignore subchar arg))
  (let* ((current-package *package*)
         (local-nicknames (package-local-nicknames current-package)))
    (loop for (nick . package) in local-nicknames
          do (remove-package-local-nickname nick current-package))
    (unwind-protect
      (read stream t nil t)
      (loop for (nick . package) in local-nicknames
            do (add-package-local-nickname nick package current-package)))))

(set-dispatch-macro-character #\# #\` #'|#`-reader|)
\end{verbatim}
It is \emph{implementation-dependent} whether \emph{local nicknames} are actually removed from
the \emph{current package} or not.
\subsubsection{Proposal PRINT-UNREADABLY}
\label{sec:org2fb6505}
In the 5th case the symbol is printed unreadably using the \texttt{\#<} syntax:
\begin{verbatim}
#<SYMBOL IN THE SHADOWED PACKAGE FOO:BAR>
#<SYMBOL IN THE SHADOWED PACKAGE FOO::BAR>
\end{verbatim}

(Specifics are \emph{implementation-dependent}.)

If \texttt{*print-readably*} is \emph{true}, an error of type \texttt{print-not-readable} is signaled.
\subsubsection{Proposal THREE-FOUR-PACKAGE-MARKERS}
\label{sec:org17347bc}
In the 5th case the symbol is printed using the extended symbol token syntax:
\begin{verbatim}
foo:::bar   ; same as (cl:find-symbol "BAR" "FOO") in the #:KEYWORD package
foo::::bar  ; same as (cl:intern "BAR" "FOO") in #:KEYWORD package
\end{verbatim}
\subsubsection{Links}
\label{sec:orga6aaebb}
See \href{https://www.lispworks.com/documentation/HyperSpec/Body/22\_acca.htm}{CLHS 22.1.3.3.1 Package Prefixes for Symbols}.
\subsection{Issue 3 (Local nicknames effect on DEFPACKAGE, MAKE-PACKAGE and others)}
\label{sec:orga21bf4f}
\subsubsection{Description}
\label{sec:orgf5abac7}
It is not clear whether \emph{local nicknames} of the \emph{current package} should affect
the package lookup in operators that accept a package designator as an argument.
This mainly concerns \texttt{make-package} and \texttt{defpackage}.
See also \hyperref[sec:org2766d95]{Notes}.
\subsubsection{Examples}
\label{sec:org02eeee7}
\begin{verbatim}
(defpackage #:foo-a (:use) (:export #:x))
(defpackage #:foo-b (:use) (:export #:x))

(defpackage #:bar
  (:use #:cl)
  (:local-nicknames (#:foo-a #:foo-b)
                    (#:foo-b #:foo-a)))

(in-package #:bar)

(defpackage #:quux-1
  (:use #:foo-a))
(package-name (symbol-package 'quux-1::x))
; => "FOO-B"  (sbcl, ccl, acl, abcl, lispworks)
; => "FOO-A"  (ecl, clasp)

(make-package '#:quux-2 :use '(#:foo-a))
(package-name (symbol-package 'quux-2::x))
; => "FOO-B"  (sbcl, ccl, ecl, acl, abcl, clasp, lispworks)

(defpackage #:quux-3
  (:use)
  (:local-nicknames (#:foo #:foo-a)))
(let ((*package* (find-package '#:quux-3)))
  (package-name (find-package '#:foo)))
; => "FOO-B"  (ccl, ecl, acl, abcl)
; => "FOO-A"  (sbcl, clasp, lispworks)

(import (car (find-all-symbols (string '#:add-package-local-nickname))))
(defpackage #:quux-4
  (:use))
(add-package-local-nickname '#:foo '#:foo-a '#:quux-4)
(let ((*package* (find-package '#:quux-4)))
  (package-name (find-package '#:foo)))
; => "FOO-B"  (ccl, ecl, clasp, abcl, lispworks)
; => "FOO-A"  (sbcl)

(use-package '#:foo-a '#:quux-4)
(package-name (symbol-package 'quux-4::x))
; => "FOO-B"  (sbcl, ccl, ecl, clasp, abcl, lispworks)
\end{verbatim}
\subsubsection{Current behavior}
\label{sec:org430a3fc}
\begin{itemize}
\item \texttt{defpackage} and \texttt{make-package}
sbcl, lispworks:
  Only the \texttt{:local-nicknames} clause is \textbf{not} affected by \emph{local nicknames}.

ccl, acl, abcl:
  All clauses and keyword arguments are affected.

ecl:
  Only the \texttt{:local-nicknames} clause and all keyword arguments (\texttt{:use} and
  \texttt{:local-nicknames}) are affected.

clasp:
  Only the keyword argument \texttt{:use} is affected.
\item Known exceptions in other operators:
sbcl: \texttt{add-package-local-nickname} is not affected.

lispworks: \texttt{in-package} is not affected.
\end{itemize}
\subsubsection{Proposal ALL-AFFECTED}
\label{sec:orgea80085}
All operators taking a package designator as an argument must be affected by
\emph{local nicknames} of the \emph{current package}.

In particular, all \texttt{defpackage} clauses (\texttt{:use}, \texttt{:local-nicknames},
\texttt{:import-from}, \texttt{:shadowing-import-from}) as well as all keyword arguments to
\texttt{make-package} (\texttt{:use} and \texttt{:local-nicknames}) must be affected.
\subsubsection{Notes}
\label{sec:org2766d95}
A non-exhaustive list of operators possibly affected by this issue:
  \texttt{defpackage}, \texttt{make-package}, \texttt{export}, \texttt{find-symbol}, \texttt{import},
  \texttt{rename-package}, \texttt{shadow}, \texttt{shadowing-import}, \texttt{delete-package},
  \texttt{with-package-iterator}, \texttt{unexport}, \texttt{unintern}, \texttt{in-package}, \texttt{unuse-package},
  \texttt{use-package}, \texttt{do-symbols}, \texttt{do-external-symbols}, \texttt{do-all-symbols}, \texttt{intern},
  \texttt{package-name}, \texttt{package-nicknames}, \texttt{package-shadowing-symbols},
  \texttt{package-use-list}, \texttt{package-used-by-list}, \texttt{add-package-local-nickname},
  \texttt{remove-package-local-nickname}, \texttt{package-local-nicknames},
  \texttt{package-locally-nicknamed-by}.

\texttt{format} is \textbf{not} affected by this issue, see instead \hyperref[sec:orgd77e416]{Issue 8}.
\subsection{Issue 4 (Local nicknames of the package being defined)}
\label{sec:org43f3224}
\subsubsection{Description}
\label{sec:org7e9bac6}
It is not clear whether \emph{local nicknames} of the package \textbf{being defined} should
affect \texttt{make-package} or \texttt{defpackage}.
\subsubsection{Examples}
\label{sec:org86f427e}
\begin{verbatim}
(defpackage #:foo-a (:use) (:export #:x))
(defpackage #:foo-b (:use) (:export #:x))

(defpackage #:bar
  (:local-nicknames (#:foo-a #:foo-b)
                    (#:foo-b #:foo-a))
  (:use #:foo-a))

(package-name (symbol-package 'bar::x))
; => "FOO-A"  (sbcl, ccl, acl, abcl, clasp, lispworks)
; => "FOO-B"  (ecl)
\end{verbatim}
\subsubsection{Current behavior}
\label{sec:orge5c2e54}
sbcl, ccl, acl, abcl, lispworks:
  Nothing is affected.

ecl:
  \texttt{:use}, \texttt{:import-from} and \texttt{:shadowing-import-from} clauses are affected.

clasp:
  \texttt{:local-nicknames} clause is affected by \emph{local nicknames} introduced by
  previous \texttt{:local-nicknames} clauses.
\subsubsection{Proposal NO-EFFECT}
\label{sec:org3d26b4b}
Local nicknames of the package being defined must not affect any of the defpackage
clauses (\texttt{:use}, \texttt{:local-nicknames}, \texttt{:import-from}, \texttt{:shadowing-import-from}).

The keyword argument \texttt{:local-nicknames} to \texttt{make-package} must not affect the
\texttt{:use} keyword argument either.
\subsection{Issue 5 (Local nickname shadowing package's own name)}
\label{sec:org72887a2}
\subsubsection{Description}
\label{sec:orged58d1d}
It is not clear whether it should be allowed to define a \emph{local nickname} that
shadows the \emph{name} or one of the \emph{global nicknames} of the package that it is
defined in.
\subsubsection{Examples}
\label{sec:org2a986b3}
\begin{verbatim}
(defpackage #:foo
  (:use)
  (:nicknames #:bar)
  (:local-nicknames (#:foo #:cl)
                    (#:bar #:cl)))
; => continuable error  (sbcl, ccl, abcl)
; => error  (lispworks)
; => ok  (ecl, acl, clasp)
\end{verbatim}
\subsubsection{Current behavior}
\label{sec:org31dc112}
sbcl, ccl, abcl:
  A correctable error is signaled by \texttt{defpackage}.

lispworks:
  An error is signaled by \texttt{defpackage}.
  A correctable error is signaled by \texttt{add-package-local-nickname}.

ecl, acl, clasp:
  No errors are signaled.
\subsubsection{Proposal ALLOW-WITH-STYLE-WARNING}
\label{sec:orgab25805}
It should be allowed to use the name or one of the global nicknames of the package
as a local nickname, but a style warning might be issued.
\begin{enumerate}
\item Rationale
\label{sec:orgfc9fd35}
Such local nicknames are not likely to break anything. Even though they can be a
bit confusing, this alone does not warrant an error to be signaled.

Moreover, on all implementations it is possible to obtain such nicknames, even if
on some of them invoking the \texttt{continue} restart is required. This suggests that
cost of adoption is very low.
\end{enumerate}
\subsubsection{Proposal DISALLOW-WITH-CORRECTABLE-ERROR}
\label{sec:orgb9655c1}
Attemts to create such a nickname should result in a correctable error being
signaled. The \texttt{continue} restart can be used to create the nickname anyway.
\subsection{Issue 6 (Additional keyword argument to MAKE-PACKAGE)}
\label{sec:org09ee005}
\subsubsection{Descriptions}
\label{sec:org17110e3}
An additional keyword argument to \texttt{make-package} similar to the \texttt{:use} keyword
argument would be a nice feature. There are only two implementations introducing
this extension, unfortunately they use different API for this argument.

Possible questions:
\begin{itemize}
\item Does it accept an alist or a nested list?
\item Is the default value implementation-dependent, or is it an empty list?
\end{itemize}
\subsubsection{Current behavior}
\label{sec:org779a9f5}
sbcl, ccl, abcl, clasp, lispworks: no additional keyword argument.

ecl: has an additional keyword argument \texttt{:local-nicknames}, but it is undocumented
and it segfaults on incorrect usage. The expected value is a list of conses:
\texttt{((nickname . package)*)}.

acl: has an additional keyword argument \texttt{:local-nicknames}. The expected value is
a list of lists: \texttt{((nickname package)*)}.
\subsubsection{Proposal EXTRA-KEYWORD-ARGUMENT-LIST}
\label{sec:org0815925}
Add \texttt{:local-nicknames} keyword argument to \texttt{make-package}:
\begin{verbatim}
local-nicknames ::= ((nickname package)*)
\end{verbatim}


\texttt{nickname} --- a \emph{string designator}.

\texttt{package} --- a \emph{package designator}.

\texttt{local-nicknames} defaults to an \emph{empty list}.
\subsubsection{Links}
\label{sec:org7e6978a}
See \hyperref[sec:org11654fe]{make-package}.
\subsection{Issue 7 (Multiple local nicknames)}
\label{sec:org4ce525f}
\subsubsection{Description}
\label{sec:org46beb1c}
It is not clear whether \texttt{package-locally-nicknamed-by-list} should be allowed to return a
list with duplicate entries (when there are multiple local nicknames in one package).
\subsubsection{Examples}
\label{sec:orgdd10ac6}
\begin{verbatim}
(defpackage #:foo
  (:use)
  (:local-nicknames (#:bar #:cl)
                    (#:baz #:cl)))
(mapcar #'package-name (package-locally-nicknamed-by-list '#:cl))
; => ("FOO")  (sbcl, acl, clasp, lispworks)
; => ("FOO" "FOO")  (ccl, ecl, abcl)
\end{verbatim}
\subsubsection{Current behavior}
\label{sec:orgb32a3b4}
sbcl, acl, clasp, lispworks:
  \texttt{package-locally-nicknamed-by-list} never contains duplicate entries.

ccl, abcl, ecl:
  \texttt{package-locally-nicknamed-by-list} might contain duplicate entries.
\subsubsection{Proposal NO-DUPLICATES}
\label{sec:orgb29a438}
\texttt{package-locally-nicknamed-by-list} should return a list without duplicate entries.
\subsection{Issue 8 (Interaction with FORMAT)}
\label{sec:orgd77e416}
\emph{by |3b|}
\subsubsection{Description}
\label{sec:orgbb7aed0}
It is not clear how \emph{local nicknames} should affect the tilde slash \texttt{\textbackslash{}\textasciitilde{}//}
directive of \texttt{format}.
\subsubsection{Notes}
\label{sec:org0177c0e}
\begin{itemize}
\item If \texttt{\textbackslash{}\textasciitilde{}//} directive wasn't affected by \emph{local nicknames}, it would mean that
using [long and not-so-easy-to-use] package names is necessary.

\item It would be counterintuitive if the function used by the \texttt{\textbackslash{}\textasciitilde{}//} directive could
change based on the \emph{current package} at \textbf{execution time} (that is, if the
symbol lookup was affected by \emph{local nicknames} of the \emph{current package}).

This could also break existing code if a \emph{local nickname} in the \emph{current
package} shadows the name of a package containing a function used in a format
control string in the called function. The only way to prevent this issue would
be to rebind the \texttt{*package*} variable around any call to \texttt{format} where the
\texttt{\textbackslash{}\textasciitilde{}//} directive is used.

\item Finally, if the call to \texttt{format} is not compiled, it would be hard to impossible
to find the function using \emph{local nicknames} of the package that was the
\emph{current package} at \textbf{compile time}.
\end{itemize}
\subsubsection{Examples}
\label{sec:org0952e0e}
\begin{verbatim}
(defpackage #:foo-a (:use) (:export #:ff))
(defpackage #:foo-b (:use) (:export #:ff))

(defun foo-a:ff (stream &rest args)
  (declare (ignore args))
  (format stream "FOO-A:FF"))

(defun foo-b:ff (stream &rest args)
  (declare (ignore args))
  (format stream "FOO-B:FF"))

(defpackage #:bar-a
  (:use #:cl)
  (:local-nicknames (#:nick #:foo-a)))

(defpackage #:bar-b
  (:use #:cl)
  (:local-nicknames (#:nick #:foo-b)))

(in-package #:bar-a)

(defun test ()
  (format t "Called ~/nick:ff/ & " nil)
  (let ((*package* (find-package (quote #:bar-a))))  ; or #.*package*
    (format t "~/nick:ff/~%" nil)))

(test)
; => "Called FOO-A:FF & FOO-A:FF"  (sbcl, ccl, ecl, acl, abcl, clasp)
; lispworks errors (NICK package not found)
(let ((*package* (find-package (quote #:bar-b))))
  (test))
; => "Called FOO-A:FF & FOO-A:FF"  (sbcl, clasp)
; => "Called FOO-B:FF & FOO-A:FF"  (ccl, ecl, acl, abcl)
; lispworks errors (NICK package not found)
\end{verbatim}
\subsubsection{Proposal NOT-AFFECTED-BY-LOCAL-NICKNAMES}
\label{sec:org057376c}
The tilde slash \texttt{\textbackslash{}\textasciitilde{}//} directive of \texttt{format} must \textbf{not} use \emph{local nicknames} of
any package when looking up the specified symbol.
\begin{enumerate}
\item Rationale
\label{sec:org19f49a2}
When the package prefix is not specified, the specified symbol is not looked up
in the \emph{current package}, but instead in the \texttt{\#:CL-USER} package. That suggests
that the tilde slash \texttt{\textbackslash{}\textasciitilde{}//} directive should not depend on the value of
\texttt{*package*} at any time.

Note that specifying it to use \emph{local nicknames} of the \texttt{\#:CL-USER} package would
risk breaking the existing code, since it is allowed to add local nicknames to
the \texttt{\#:CL-USER} package.
\end{enumerate}
\subsubsection{Links}
\label{sec:org8738b3d}
See \href{https://www.lispworks.com/documentation/HyperSpec/Body/22\_ced.htm}{CLHS 22.3.5.4 Tilde Slash: Call Function}.
\subsection{Issue 9 (Empty string as a local name)}
\label{sec:org04a331d}
\subsubsection{Description}
\label{sec:org9054330}
It is not clear whether it should be allowed to use the empty string \texttt{""} as a
local nickname, and in the case it is allowed, whether it should affect the
keyword symbol syntax \texttt{:xxxx}.
\subsubsection{Examples}
\label{sec:org953f274}
\begin{verbatim}
(defpackage #:foo
  (:use #:cl)
  (:local-nicknames ("" #:cl)))

(in-package #:foo)

(package-name (symbol-package ':*package*))
; => "KEYWORD"  (sbcl, ccl, ecl, abcl, clasp, lispworks)
; => "COMMON-LISP"  (acl)

(package-name (symbol-package '||:*package*))
; => "KEYWORD"  (ecl, clasp, lispworks)
; => "COMMON-LISP"  (sbcl, ccl, acl)
; abcl errors
\end{verbatim}
\subsubsection{Current behavior}
\label{sec:org5cb2c3b}
sbcl, ccl:
\texttt{:xxxx} is read as a keyword;
\texttt{||:xxxx} is read as a symbol in the package named or nicknamed \texttt{""}.

ecl, clasp, lispworks:
Both \texttt{||:xxxx} and \texttt{:xxxx} are read as a keyword.

acl:

Both \texttt{:xxxx} and \texttt{||:xxxx} are read as a symbol in the package named or nicknamed
\texttt{""} which is a \emph{global nickname} for the \texttt{\#:KEYWORD} package, but can be shadowed
by a \emph{local nickname}.

abcl:
\texttt{:xxxx} is read as a keyword;
\texttt{||:xxxx} syntax cannot be read (attemts result in an error).
\subsubsection{Proposal ALLOW-KEEP-KEYWORDS}
\label{sec:orgab42497}
The \texttt{""} local nickname should be explicitely allowed. \texttt{:xxxx} should be always
read as a keyword regardless of package names or nicknames. \texttt{||:xxxx} should be
read as a symbol in the package named or nicknamed by \texttt{""}.
\subsubsection{Proposal ALLOW-FUN}
\label{sec:orgc63633e}
The \texttt{""} local nickname should be explicitely allowed. Both \texttt{:xxxx} and \texttt{||:xxxx}
should be read as a symbol in the package named or nicknamed by \texttt{""}.
\subsubsection{Links}
\label{sec:org09859b4}
See also the \href{https://github.com/s-expressionists/wscl/issues/63}{WSCL issue 63}.
\section{Links}
\label{sec:orgdab25c9}
3b's \href{https://github.com/3b/package-local-nicknames/blob/master/docs.org}{notes} on package-local nicknames.

phoe's \href{https://github.com/phoe/trivial-package-local-nicknames}{tests}.

SBCL's \href{https://www.sbcl.org/manual/\#Package\_002dLocal-Nicknames}{manual entry}.

Section 4.3 of the ABCL's manual. (\href{https://github.com/armedbear/abcl/blob/master/doc/manual/abcl.tex\#L1249}{\TeX{} file on github})

Allegro CL's \href{https://franz.com/support/documentation/11.0/packages.html\#local-nicknames-1}{documentation}.
\section{Copying and License}
\label{sec:org8ebe09c}
[TODO]
\end{document}