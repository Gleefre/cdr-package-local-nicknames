% Created 2024-07-06 Sat 13:21
% Intended LaTeX compiler: pdflatex
\documentclass[11pt]{article}
\usepackage[utf8]{inputenc}
\usepackage[T1]{fontenc}
\usepackage{graphicx}
\usepackage{longtable}
\usepackage{wrapfig}
\usepackage{rotating}
\usepackage[normalem]{ulem}
\usepackage{amsmath}
\usepackage{amssymb}
\usepackage{capt-of}
\usepackage{hyperref}
\usepackage[margin=1in]{geometry}
\author{Gleefre}
\date{\today}
\title{PLN CDR draft: spec}
\hypersetup{
 pdfauthor={Gleefre},
 pdftitle={PLN CDR draft: spec},
 pdfkeywords={},
 pdfsubject={},
 pdfcreator={Emacs 28.2 (Org mode 9.5.5)}, 
 pdflang={English}}
\begin{document}

\maketitle

\section{Introduction}
\label{sec:orgb1e1bd5}
This is a specification for the \texttt{Package-Local Nicknames} extension in Common Lisp.
\subsection{Rationale}
\label{sec:org7058bee}
Package-local nicknames make it possible to use short and easy-to-use names
without potentially introducing name conflicts as can happen with usual nicknames.
\subsection{Current state}
\label{sec:org53d26c0}
Package-local nicknames are implemented in some form in \texttt{SBCL}, \texttt{CCL}, \texttt{ECL},
\texttt{Clasp}, \texttt{ABCL}, \texttt{Allegro CL}, \texttt{LispWorks}. There is also a pending MR for the
\texttt{CLISP} implementation.

Unfortunately, there are multiple inconsistencies between implementations. All of
them lose \emph{print-read consistency} to some extent, and there are multiple edge
cases that aren't always implemented correctly or in the same way.
\subsection{Goal}
\label{sec:org87ec8e6}
The purpose of this document is to standardize the \texttt{Package-Local Nicknames}
extension and to address some existing issues.

[TODO] This CDR also aims to provide an extensive test suite for the extension.
\section{Description}
\label{sec:orgb3c90cf}
A \emph{package-local nickname} (or a \emph{local nickname}) defined in some \emph{designated
package} has the same effects as a usual \emph{package nickname} (later referred to as a
\emph{global nickname}), except that these effects only apply when \texttt{*package*} is bound
to that \emph{designated package}.

This means that a call to \texttt{find-package} with a \emph{local nickname} that is defined in
the \emph{current package} returns the package nicknamed by this nickname. This also
affects all implied calls to \texttt{find-package}, including those performed by the Lisp
reader.

In addition, to maintain \emph{print-read consistency}, the Lisp printer is affected by
\emph{local nicknames} defined in the \emph{current package}.  For details see Issue 2.

A \emph{local nickname} is allowed to shadow a \emph{package name} or a \emph{global nickname},
except for the names \texttt{\#:CL}, \texttt{\#:COMMON-LISP} and \texttt{\#:KEYWORD} which must always
refer to their packages. The consequences of adding \emph{local nicknames} to the
packages \texttt{\#:COMMON-LISP} and \texttt{\#:KEYWORD} are also undefined.
\section{API}
\label{sec:org84a03a1}
\subsection{defpackage}
\label{sec:org1f38869}
\subsubsection{Description}
\label{sec:orgf7ce34b}
The \texttt{defpackage} options are extended to include the \emph{local-nicknames-option}:
\begin{verbatim}
local-nicknames-option ::= (:local-nicknames (nickname package)*)
\end{verbatim}


Each pair specifies a \emph{local nickname} \texttt{nickname} for the corresponding \texttt{package}.

This option may appear more than once.
\subsubsection{Arguments and Values:}
\label{sec:orgd65d238}
\texttt{nickname} --- a \emph{string designator}.

\texttt{package} --- a \emph{package designator}.
\subsubsection{Exceptional situations}
\label{sec:org63798b8}
An error of type \texttt{package-error} is signaled when a package designated by the
\texttt{package} does not exists.

Name conflict errors are handled by the underlying calls to
\texttt{add-package-local-nickname}.

See \hyperref[sec:orgdcd5ecd]{add-package-local-nickname: exceptional situations}.
\subsubsection{Implementation dependent}
\label{sec:org6f8fe2a}
The consequences are undefined when a \emph{local nickname} is specified for thex
package that is being defined. (See Issue 4.)

The consequences are undefined when supplied \emph{local nicknames} are at variance
with the current state of the package. An implementation might choose to remove
all existing \emph{local nicknames} at the beginning of each redefinition of the
package.
\subsection{make-package}
\label{sec:orgf3a0730}
\subsubsection{Description}
\label{sec:org8581069}
(\textbf{Contains proposals}: see Issue 6.)

The \texttt{make-package} lambda list is extended to include an additional keyword
argument \texttt{:local-nicknames}:
\begin{verbatim}
local-nicknames ::= ((nickname package)*)
\end{verbatim}


\texttt{local-nicknames} specifies zero or more \emph{local nicknames} to be defined in the
new \emph{package}.
\subsubsection{Arguments and Values:}
\label{sec:orgc7abff5}
\texttt{local-nicknames} --- a \emph{list} of pairs of form \texttt{(nickname package)}.
The default is an \emph{empty list}.

\texttt{nickname} --- a \emph{string designator}.

\texttt{package} --- a \emph{package designator}.
\subsubsection{Exceptional situations}
\label{sec:org4d7c11f}
An error of type \texttt{package-error} is signaled when a package designated by the
\texttt{package} does not exist.

Name conflict errors are handled by the underlying calls to
\texttt{add-package-local-nickname}.

See \hyperref[sec:orgdcd5ecd]{add-package-local-nickname: exceptional situations}.
\subsubsection{Implementation dependent}
\label{sec:org2696bbb}
The consequences are undefined when a \emph{local nickname} is specified for the
package that is being defined. (See Issue 4.)
\subsection{add-package-local-nickname}
\label{sec:orgceafe06}
\begin{verbatim}
(add-package-local-nickname nickname actual-package &optional designated-package)
  => designated-package-object
\end{verbatim}
\subsubsection{Arguments and Values}
\label{sec:orgb9143c0}
\texttt{nickname} --- a \emph{string designator}.

\texttt{actual-package} --- a \emph{package designator}.

\texttt{designated-package} --- a \emph{package designator}.
The default is the \emph{current package}.

\texttt{designated-package-object} --- a \emph{package}.
\subsubsection{Description}
\label{sec:org4a177ba}
Defines a \emph{package-local nickname} \texttt{nickname} for the \texttt{actual-package} in the
\texttt{designated-package}.

[Also see Issue 1.] Returns the package designated by the \texttt{designated-package}.

If the \emph{nickname} is already defined, checks that it is defined for the package
designated by the \texttt{actual-package}.
\subsubsection{Exceptional situations}
\label{sec:orgdcd5ecd}
An error of type \emph{package-error} is signaled when a package designated by the
\texttt{actual-package} or the \texttt{designated-package} does not exist.

If the \texttt{nickname} is one of the names \texttt{\#:CL}, \texttt{\#:COMMON-LISP} or \texttt{\#:KEYWORD}, an
error of type \emph{package-error} is signaled.

If the \texttt{nickname} is already defined to be a \emph{local nickname} for a package
different from the \texttt{actual-package}, a \emph{correctable} error of type \emph{package-error}
is signaled.
\subsubsection{Implementation dependent}
\label{sec:org39f6a11}
The consequences are undefined when the \texttt{designated-package} designates the
\texttt{\#:COMMON-LISP} package or the \texttt{\#:KEYWORD} package.

(\textbf{Contains proposals}: see Issue 5.)

If the \texttt{nickname} shadows the \emph{package name} or one of the \emph{global nicknames} of
the \texttt{designated-package}, a style warning might be issued.
\subsection{remove-package-local-nickname}
\label{sec:org4afd585}
\begin{verbatim}
(remove-package-local-nickname old-nickname &optional designated-package)
  => nickname-removed-p
\end{verbatim}
\subsubsection{Arguments and Values}
\label{sec:orga81c135}
\texttt{old-nickname} --- a \emph{string designator}.

\texttt{designated-package} --- a \emph{package designator}.
The default is the \emph{current package}.

\texttt{nickname-removed-p} --- \emph{generalized boolean}.
\subsubsection{Description}
\label{sec:orgefda88a}
If \texttt{old-nickname} is defined to be a \emph{local nickname} in the \texttt{designated-package},
it is removed.

[Also see Issue 1.] Returns \emph{true} if it removes a nickname, and \texttt{NIL} otherwise.
\subsubsection{Exceptional situations}
\label{sec:orgcb4f7cc}
An error of type \emph{package-error} is signaled when a package designated by the
\texttt{designated-package} does not exist.
\subsection{package-local-nicknames}
\label{sec:org35164bb}
\begin{verbatim}
(package-local-nicknames package-designator)
  => local-nicknames-alist
local-nicknames-alist ::= ((nickname . package)*)
\end{verbatim}
\subsubsection{Arguments and Values}
\label{sec:orgbdb16cd}
\texttt{package-designator} --- a \emph{package designator}.

\texttt{local-nicknames-alist} --- an \emph{alist}.

\texttt{nickname} --- a \emph{string}.

\texttt{package} --- a \emph{package}.
\subsubsection{Description}
\label{sec:org347322c}
Returns an \emph{alist} describing \emph{local nicknames} defined in the package designated
by the \texttt{package-designator}.
\subsubsection{Exceptional situations}
\label{sec:orgca1944b}
An error of type \texttt{package-error} is signaled when a package designated by the
\texttt{package-designator} does not exist.
\subsubsection{Notes}
\label{sec:orgd4382e7}
The returned \emph{alist} must be safe to be modified by the user.
\subsection{package-locally-nicknamed-by-list}
\label{sec:orgb877bc1}
\begin{verbatim}
(package-locally-nicknamed-by-list package-designator)
  => packages-list
\end{verbatim}
\subsubsection{Arguments and Values}
\label{sec:org980f4cf}
\texttt{package-designator} --- a \emph{package designator}.

\texttt{packages-list} --- a \emph{list} of \emph{package} objects.
\subsubsection{Description}
\label{sec:orgb29b47f}
Returns a \emph{list} of packages that have a \emph{local nickname} defined for the package
designated by the \texttt{package-designator}.
\subsubsection{Exceptional situations}
\label{sec:orga2a1f07}
An error of type \texttt{package-error} is signaled when a package designated by the
\texttt{package-designator} does not exist.
\subsubsection{Notes}
\label{sec:org3897f98}
The returned \emph{list} must be safe to be modified by the user.
\section{Affected symbols}
\label{sec:orgaa7116e}
\subsection{defpackage}
\label{sec:org94b3bdb}
See \hyperref[sec:org1f38869]{defpackage}.
\subsection{make-package}
\label{sec:orga8c1e02}
See \hyperref[sec:orgf3a0730]{make-package}.
\subsection{find-package}
\label{sec:org0ed4c47}
(\textbf{Contains proposals}: see Issue 3, Issue 8.)

When the argument to \texttt{find-package} is a \emph{local nickname} defined in the \emph{current
package}, it returns the package nicknamed by this nickname.

This also affects all implied calls to \texttt{find-package}, including but not limited
to those performed by the lisp reader as well as those performed by \texttt{defpackage},
\texttt{make-package}, \texttt{export}, \texttt{find-symbol}, \texttt{import}, \texttt{rename-package}, \texttt{shadow},
\texttt{shadowing-import}, \texttt{delete-package}, \texttt{with-package-iterator}, \texttt{unexport},
\texttt{unintern}, \texttt{in-package}, \texttt{unuse-package}, \texttt{use-package}, \texttt{do-symbols},
\texttt{do-external-symbols}, \texttt{do-all-symbols}, \texttt{intern}, \texttt{package-name},
\texttt{package-nicknames}, \texttt{package-shadowing-symbols}, \texttt{package-use-list},
\texttt{package-used-by-list}.

\texttt{add-package-local-nickname}, \texttt{remove-package-local-nickname},
\texttt{package-local-nicknames} and \texttt{package-locally-nicknamed-by} are also affected.

The only exception is the \emph{tilde slash} directive of \texttt{format}, which should \textbf{not}
use \emph{local nicknames} from any package when looking up the specified symbol.
\subsection{rename-package}
\label{sec:orga809406}
When a package is renamed with \texttt{rename-package}, it retains all \emph{local nicknames}
it has defined, as well as all \emph{local nicknames} by which it is nicknamed.

\subsubsection{Implementation dependent}
\label{sec:orge5d4e8b}
(\textbf{Contains proposals}: see Issue 5.)

If the \emph{new-name} or one of the \emph{new-nicknames} is shadowed by one of the \emph{local
nicknames} of the package being renamed, a style warning might be issued.
\subsection{delete-package}
\label{sec:org670a32e}
When a package is deleted with \texttt{delete-package}, all \emph{local nicknames} defined in
that package are removed, as well as all \emph{local nicknames} by which it is
nicknamed.

This also means that a deleted package must not be available via calls to
\texttt{package-locally-nicknamed-by-list} and \texttt{package-local-nicknames}.
\subsection{format}
\label{sec:orgfcde85d}
See Issue 8.
\subsection{$\backslash$*features$\backslash$*}
\label{sec:org14e8394}
If an implementation supports package-local nicknames, it should add symbols
\texttt{:package-local-nicknames} and \texttt{:cdr-NN} (per CDR 14) to \texttt{*features*}.
\section{Examples}
\label{sec:org2022a71}
[TODO]
\end{document}