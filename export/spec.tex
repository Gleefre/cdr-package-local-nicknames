% Created 2024-07-05 Fri 09:32
% Intended LaTeX compiler: pdflatex
\documentclass[11pt]{article}
\usepackage[utf8]{inputenc}
\usepackage[T1]{fontenc}
\usepackage{graphicx}
\usepackage{longtable}
\usepackage{wrapfig}
\usepackage{rotating}
\usepackage[normalem]{ulem}
\usepackage{amsmath}
\usepackage{amssymb}
\usepackage{capt-of}
\usepackage{hyperref}
\usepackage[margin=1in]{geometry}
\author{Grolter Bell}
\date{\today}
\title{}
\hypersetup{
 pdfauthor={Grolter Bell},
 pdftitle={},
 pdfkeywords={},
 pdfsubject={},
 pdfcreator={Emacs 28.2 (Org mode 9.5.5)}, 
 pdflang={English}}
\begin{document}


\section{Introduction}
\label{sec:orgb428c72}
This is a specification for the package-local nicknames extension in Common Lisp.
\subsection{Rationale}
\label{sec:org8ee42a4}
Package-local nicknames allow to use short and easy-to-use names for packages
locally without potentially introducing name conflict as with normal nicknames.
\subsection{Current state}
\label{sec:org6299332}
Package-local nicknames are implemented in some form in \texttt{SBCL}, \texttt{CCL}, \texttt{ECL},
\texttt{Clasp}, \texttt{ABCL}, \texttt{Allegro CL}, \texttt{LispWorks}. There also exists a pending PR for the
\texttt{CLISP} implementation.

Unfortunately, there are multiple inconsistencies between implementations, all
implementations lose the \textbf{print-read} consistency to some extent, and there are
multiple edge cases that aren't always implemented correctly.
\subsection{Goal}
\label{sec:org4810ec2}
The purpose of this document is to standardize the package-local nicknames
extension and to address some existing issues.

[TODO] This CDR also aims to provide an extensive test suite for this extension.
\section{Description}
\label{sec:orga1b49e2}
\emph{Package-local nickname} (or \emph{local nickname}) has the same effects as a
normal \emph{package nickname} (later \emph{global nickname}), except that these
effects only apply when \texttt{*package*} is bound to a package for which the
nickname has been defined.

That means that calls to \texttt{find-package} with a \emph{local nickname} defined in
the \emph{current package} should return the package nicknamed by this nickname.

This also affects all implied calls to \texttt{find-package}, including those
performed by the lisp reader.

In addition, to maintain \textbf{print-read} consistency, the lisp printer is
affected by \emph{local nicknames} defined in the \emph{current package}, for details
see PRINT-READ consistency.

\emph{Local nickname} is allowed to shadow a \emph{package name} or a \emph{global
nickname}, except for the names \texttt{\#:CL}, \texttt{\#:COMMON-LISP} and \texttt{\#:KEYWORD}
which should always refer to their packages.

The consequences of adding \emph{local nicknames} to the packages
\texttt{\#:COMMON-LISP} and \texttt{\#:KEYWORD} are undefined.
\section{API}
\label{sec:org613db5c}
\subsection{defpackage}
\label{sec:org393c69d}
\texttt{defpackage} options are extended to include \emph{local-nicknames-option}:
\begin{verbatim}
local-nicknames-option ::= (:local-nicknames (nickname package)*)
\end{verbatim}


Each pair specifies a \emph{local nickname} \texttt{nickname} for the corresponding
\texttt{package}.

This option may appear more than once.
\subsubsection{Arguments and Values:}
\label{sec:org5a306c1}
\texttt{nickname} must be a \emph{string designator}.

\texttt{package} must be a \emph{package designator}.
\subsubsection{Exceptional situations}
\label{sec:orgc40c1e8}
An error of type \texttt{package-error} is signaled when a package designated by
\texttt{package} does not exists.

Name conflict errors are handled by the underlying calls to
\texttt{add-package-local-nickname}.

See \hyperref[sec:org05d592d]{add-package-local-nickname: exceptional situations}.
\subsubsection{Implementation dependent}
\label{sec:org7ca6297}
The behaviour is unspecified when a \emph{local nickname} is specified for the
package that is being defined.

The behaviour is unspecified when supplied \emph{local nicknames} are at
variance with the current state of the package that is being defined. An
implementation might choose to remove all present \emph{local nicknames} at
the begining of each redefinition of the package.
\subsection{make-package}
\label{sec:org662ac14}
\texttt{make-package} lambda list is extended to include an additional key
parameter: \texttt{local-nicknames}.
\begin{verbatim}
local-nicknames ::= ((nickname package)*)
\end{verbatim}


\texttt{local-nicknames} defaults to an \emph{empty list}.

\texttt{local-nicknames} must be a \emph{list} each element of which must be a \emph{list}
of form \texttt{(nickname package)}. Specifies \emph{local nicknames} in the new
\emph{package}.

See also Issue 6.
\subsubsection{Arguments and Values:}
\label{sec:orga31756a}
\texttt{local-nicknames} must be a a \emph{list} of pairs \texttt{(nickname package)}.

\texttt{nickname} must be a \emph{string designator}.

\texttt{package} must be a \emph{package designator}.
\subsubsection{Exceptional situations}
\label{sec:orgb9d5456}
An error of type \texttt{package-error} is signaled when a package designated by
\texttt{package} does not exists.

Name conflict errors are handled by the underlying calls to
\texttt{add-package-local-nickname}.

See \hyperref[sec:org05d592d]{add-package-local-nickname: exceptional situations}.
\subsubsection{Implementation dependent}
\label{sec:org2faa94f}
The behaviour is unspecified when a \emph{local nickname} is specified for the
package that is being defined.
\subsection{add-package-local-nickname}
\label{sec:org9193f50}
\begin{verbatim}
(add-package-local-nickname nickname actual-package &optional designated-package)
  => designated-package-object
\end{verbatim}

\texttt{designated-package} defaults to the \emph{current package}.

Adds a \emph{package-local nickname} \texttt{nickname} for the \texttt{actual-package} in the
\texttt{designated-package}.

Returns the package designated by \texttt{designated-package}.

If a \emph{nickname} is already defined, checks that it is defined for the
package designated by \texttt{actual-package}.
\subsubsection{Arguments and Values}
\label{sec:org0489025}
\texttt{nickname} must be a \emph{string designator}.

\texttt{actual-package} and \texttt{designated-package} must be \emph{package designators}.

\texttt{designated-package-object} is of type \emph{package}.
\subsubsection{Exceptional situations}
\label{sec:org05d592d}
If a package designated by \texttt{actual-package} or a package designated by
\texttt{designated-package} does not exists, an error of type \emph{package-error}
must be signaled.

If \texttt{nickname} is one of the names \texttt{\#:CL}, \texttt{\#:COMMON-LISP} or \texttt{\#:KEYWORD},
an error of type \emph{package-error} must be signaled.

If \texttt{nickname} is a \emph{local nickname} for a package different from
\texttt{actual-package}, an error of type \emph{package-error} must be signaled.
\subsubsection{Implementation dependent}
\label{sec:orgfc9ced8}
\textbf{PROPOSAL} (See Issue 4.)

If \texttt{nickname} shadows the \texttt{designated-package}'s \emph{package name} or one of
its \emph{global nicknames}, a style warning might signaled.
\subsection{remove-package-local-nickname}
\label{sec:org3b628fe}
\begin{verbatim}
(remove-package-local-nickname old-nickname &optional designated-package)
  => nickname-removed-p
\end{verbatim}

\texttt{designated-package} defaults to the \emph{current package}.

If \texttt{designated-package} has \texttt{old-nickname} as a \emph{local nickname}, it is
removed.

Returns \emph{true} if the \texttt{old-nickname} existed (and was removed), and \texttt{NIL}
otherwise.
\subsubsection{Arguments and Values}
\label{sec:orgf9232c3}
\texttt{old-nickname} must be a \emph{string designator}.

\texttt{designated-package} must be a \emph{package designator}.

\texttt{nickname-removed-p} is a \emph{generalized boolean}.
\subsubsection{Exceptional situations}
\label{sec:org4f0c62c}
If a package designated by \texttt{designated-package} does not exists, an error of
type \emph{package-error} must be signaled.
\subsection{package-local-nicknames}
\label{sec:org5016aae}
\begin{verbatim}
(package-local-nicknames package)
  => local-nicknames-alist
\end{verbatim}

Returns an \emph{alist} describing local nicknames defined in a package
designated by \texttt{package}.

Each cons cell in \texttt{local-nicknames-alist} is of the form \texttt{(nickname . package)}
where \texttt{nickname} is of type \emph{string} and \texttt{package} is of type
\emph{package}.
\subsubsection{Arguments and Values}
\label{sec:org79c4226}
\texttt{package} must be a \emph{package designator}.

\texttt{local-nicknames-alist} is an \emph{alist} with keys of type \emph{string} and
values of type \emph{package}.
\subsubsection{Exceptional situations}
\label{sec:org6d9b543}
An error of type \texttt{package-error} is signaled when a package designated by
\texttt{package} does not exists.
\subsubsection{Notes}
\label{sec:org54d550a}
The returned \emph{alist} must be safe to be modified by the user.
\subsection{package-locally-nicknamed-by-list}
\label{sec:org2a80039}
\begin{verbatim}
(package-locally-nicknamed-by-list package)
  => packages-list
\end{verbatim}

Returns a \emph{list} of packages that have a \emph{local nickname} defined for the
package designated by \texttt{package}.
\subsubsection{Arguments and Values}
\label{sec:orgdbb44b6}
\texttt{package} must be a \emph{package designator}.

\texttt{packages-list} is a \emph{list} with elements of type \emph{package}.
\subsubsection{Exceptional situations}
\label{sec:org459731b}
An error of type \texttt{package-error} is signaled when a package designated by
\texttt{package} does not exists.
\subsubsection{Notes}
\label{sec:org11e220f}
The returned \emph{list} must be safe to be modified by the user.
\section{Affected symbols}
\label{sec:org54d1856}
\subsection{defpackage}
\label{sec:orgf225e42}
See \hyperref[sec:org393c69d]{defpackage}.
\subsection{make-package}
\label{sec:org8f714e5}
See \hyperref[sec:org662ac14]{make-package}.
\subsection{find-package}
\label{sec:orgac33fab}
When argument to \texttt{find-package} is a \emph{local nickname} that is defined in
the \emph{current package}, it returns the package named by this nickname.

This also affects all implied calls to \texttt{find-package}, including but not
limited to those performed by the lisp reader as well as those performed
by \texttt{export}, \texttt{find-symbol}, \texttt{import}, \texttt{rename-package}, \texttt{shadow},
\texttt{shadowing-import}, \texttt{delete-package}, \texttt{with-package-iterator}, \texttt{unexport},
\texttt{unintern}, \texttt{in-package}, \texttt{unuse-package}, \texttt{use-package}, \texttt{do-symbols},
\texttt{do-external-symbols}, \texttt{do-all-symbols}, \texttt{intern}, \texttt{package-name},
\texttt{package-nicknames}, \texttt{package-shadowing-symbols}, \texttt{package-use-list},
\texttt{package-used-by-list}.

\texttt{add-package-local-nickname}, \texttt{remove-package-local-nickname},
\texttt{package-local-nicknames} and \texttt{package-locally-nicknamed-by} are also
affected.

There are two exceptions: \texttt{make-package} and \texttt{defpackage} must \textbf{not} be
affected by \emph{local nicknames} of the \emph{current package}.
\subsection{rename-package}
\label{sec:org6c442c4}
When a package is renamed via \texttt{rename-package} it maintains all \emph{local
nicknames} it is nicknamed by, as well as all \emph{local nicknames} it has
defined.
\subsubsection{Implementation dependent}
\label{sec:org7d842a0}
\textbf{PROPOSAL} (See Issue 4.)

If a \emph{new-name} or one of \emph{new-nicknames} is shadowed by one of the \emph{local
nicknames} of the package being redefined, a warning might be signaled.
\subsection{delete-package}
\label{sec:org7a57668}
When a package is deleted via \texttt{delete-package} all \emph{local nicknames}
defined in other packages that it was nicknamed by must be removed, as well
as all \emph{local nicknames} defined in the package that is being deleted.

This also means that a deleted package must not be available by calls to
\texttt{package-locally-nicknamed-by-list} and \texttt{package-local-nicknames}.
\subsection{$\backslash$*features$\backslash$*}
\label{sec:org5710d28}
If an implementation supports package-local nicknames it should add symbols
\texttt{:package-local-nicknames} and \texttt{:cdr-15} (per CDR 14) to \texttt{*features*}.
\section{Examples}
\label{sec:org830afc7}
[TODO]
\end{document}