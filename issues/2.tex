% Created 2023-06-23 Fri 18:56
% Intended LaTeX compiler: pdflatex
\documentclass[11pt]{article}
\usepackage[utf8]{inputenc}
\usepackage[T1]{fontenc}
\usepackage{graphicx}
\usepackage{grffile}
\usepackage{longtable}
\usepackage{wrapfig}
\usepackage{rotating}
\usepackage[normalem]{ulem}
\usepackage{amsmath}
\usepackage{textcomp}
\usepackage{amssymb}
\usepackage{capt-of}
\usepackage{hyperref}
\author{Grolter Bell}
\date{\today}
\title{}
\hypersetup{
 pdfauthor={Grolter Bell},
 pdftitle={},
 pdfkeywords={},
 pdfsubject={},
 pdfcreator={Emacs 27.1 (Org mode 9.3)}, 
 pdflang={English}}
\begin{document}

\tableofcontents

\section{Issue 2 (PRINT-READ consistency)}
\label{sec:orgb3a90e8}
Lisp reader uses \texttt{find-package} to read a symbol, and is affected by \emph{local
nicknames} of the \emph{current package}. So in order to maintain \textbf{print-read}
consistency it is required to use a correct \emph{package prefix} - such prefix
that calling \texttt{find-package} on it in the \emph{current package} will return the
symbol's \emph{home package}.

There are several situations to consider:
\begin{enumerate}
\item There \textbf{is} a \emph{local nickname} defined in the \emph{current package} for the
symbol's \emph{home package}.

\emph{In this case such local nickname can be used as the package prefix.}
\item Symbol's home \emph{package name} or one of its \emph{global nicknames} is not
shadowed by any \emph{local nickname} defined in the \emph{current package}.

\emph{In this case that package name or global nickname can be used as the
package prefix.}
\item Symbol's home \emph{package name} and all its \emph{global nicknames} are shadowed by
one of the \emph{local nicknames} of the \emph{current package} and there \textbf{is no}
\emph{local nickname} defined (in the \emph{current package}) for the symbol's home
package.

\emph{It is not clear what should be done in this case (see proposals).}
\end{enumerate}
\subsection{Example}
\label{sec:org277223f}
\begin{verbatim}
(defpackage #:a (:use) (:export #:+))
(defpackage #:b (:local-nicknames (#:a #:cl)))
(let ((*package* (find-package '#:b)))
  (print 'a:+))
; => a:+ everywhere
;; but in #:b package a:+ would refer to cl:+
\end{verbatim}
\begin{verbatim}
(defpackage #:a (:use) (:export #:+))
(defpackage #:b (:use) (:export #:+))
(defpackage #:c (:use) (:local-nicknames (#:a #:b) (#:b #:a)))
(let ((*package* (find-package '#:c)))
  (print 'a:+))
; => b:+  (sbcl, ccl, abcl)
; => a:+  (clasp, acl, ecl)
;; but in #:c package a:+ would refer to b:+
\end{verbatim}
\subsection{Currently}
\label{sec:org209760c}
sbcl, ccl, abcl: print symbol with package name when all package's names
and nicknames are shadowed by \emph{current package}'s \emph{local nicknames}.

ecl, acl, clasp: don't print with local nicknames at all.
\subsection{Proposals}
\label{sec:org4245d1e}
\begin{itemize}
\item The symbol must be printed using the \texttt{\#.} syntax:
\begin{verbatim}
#.(cl:let ((cl:*package* (cl:find-package "KEYWORD")))
    (cl:find-symbol "BAR" "FOO"))
;; or
#.(cl:let ((cl:*package* (cl:find-package "KEYWORD")))
    (cl:intern "BAR" "FOO"))
\end{verbatim}
Note that \texttt{\#:KEYWORD} name is reserved for the \texttt{\#:KEYWORD} package and
cannot be used as a \emph{local nickname} thus this expression will always
evaluate to the symbol \texttt{foo::bar}.
\item \emph{Shinmera's idea}. In this case an extended \texttt{\#:} syntax should be used:
\begin{verbatim}
#:(package name) and #::(package name)
\end{verbatim}
\item In this case the symbol must be printed using the \texttt{\#`} syntax for reading
an expression ignoring \emph{local nicknames} in the \emph{current package}:
\begin{verbatim}
#`foo:bar  and  #`foo::bar
\end{verbatim}


It can be implemented roughly as follows:
\begin{verbatim}
(defun |#`-reader| (stream subchar arg)
  (declare (ignore subchar arg))
  (let* ((current-package *package*)
         (local-nicknames (package-local-nicknames current-package)))
    (loop for (nick . package) in local-nicknames
          do (remove-package-local-nickname nick current-package))
    (unwind-protect
      (read stream t nil t)
      (loop for (nick . package) in local-nicknames
            do (add-package-local-nickname nick package current-package)))))

(set-dispatch-macro-character #\# #\` #'|#`-reader|)
\end{verbatim}
It is implementation dependent whether \emph{local nicknames} are actually
removed from the \emph{current package} or not.
\item In this case the symbol must be printed unreadably (specifics are
implementation dependent):
\begin{verbatim}
#<SYMBOL IN THE SHADOWED PACKAGE FOO:BAR>
#<SYMBOL IN THE SHADOWED PACKAGE FOO::BAR>
\end{verbatim}

If \texttt{*print-readably*} is \emph{true} must signal an error of type
\texttt{print-not-readable} without printing anything.
\item In this case the symbol must be printed using \texttt{:::} and \texttt{::::} syntax
to lookup and intern ignoring \emph{local nicknames} respectively:
\begin{verbatim}
foo:::bar  ; same as (cl:find-symbol "BAR" "FOO") in the #:KEYWORD package
foo::::bar  ; same as (cl:intern "BAR" "FOO") in #:KEYWORD package
\end{verbatim}
\end{itemize}
\end{document}
