% Created 2023-06-23 Fri 19:41
% Intended LaTeX compiler: pdflatex
\documentclass[11pt]{article}
\usepackage[utf8]{inputenc}
\usepackage[T1]{fontenc}
\usepackage{graphicx}
\usepackage{grffile}
\usepackage{longtable}
\usepackage{wrapfig}
\usepackage{rotating}
\usepackage[normalem]{ulem}
\usepackage{amsmath}
\usepackage{textcomp}
\usepackage{amssymb}
\usepackage{capt-of}
\usepackage{hyperref}
\author{Grolter Bell}
\date{\today}
\title{}
\hypersetup{
 pdfauthor={Grolter Bell},
 pdftitle={},
 pdfkeywords={},
 pdfsubject={},
 pdfcreator={Emacs 27.1 (Org mode 9.3)}, 
 pdflang={English}}
\begin{document}


\section{Issue 8 \emph{by |3b|}}
\label{sec:orgbed07e2}
How PLN affect \texttt{format}'s \texttt{\textbackslash{}\textasciitilde{}//} directive? It seems tricky - compilers might
want optimize it at compile time, but reffered symbol might change on
rebinding the \texttt{*package*}.
\subsection{Example}
\label{sec:org2e67018}
\begin{verbatim}
(defpackage #:a (:use) (:export #:ff))
(defpackage #:b (:use) (:export #:ff))

(defun a:ff (stream &rest args)
  (declare (ignore args))
  (format stream "A:FF"))

(defun b:ff (stream &rest args)
  (declare (ignore args))
  (format stream "B:FF"))

(defpackage #:foo
  (:use #:cl)
  (:local-nicknames (#:nick #:a)))

(defpackage #:bar
  (:use #:cl)
  (:local-nicknames (#:nick #:b)))

(in-package #:foo)
(defun test ()
  (format t "Called ~/nick:ff/ & " nil)
  (let ((*package* (find-package '#:foo)))  ; or #.*package*
    (format t "~/nick:ff/~%" nil)))

(test)
; => "Called A:FF & A:FF"  (sbcl, ccl, acl, abcl, ecl, clasp)

(let ((*package* (find-package '#:bar)))
  (test))
; => "Called A:FF & A:FF"  (sbcl, clasp)
; => "Called B:FF & A:FF"  (ccl, acl, abcl, ecl)
\end{verbatim}
\end{document}
