% Created 2023-06-23 Fri 18:51
% Intended LaTeX compiler: pdflatex
\documentclass[11pt]{article}
\usepackage[utf8]{inputenc}
\usepackage[T1]{fontenc}
\usepackage{graphicx}
\usepackage{grffile}
\usepackage{longtable}
\usepackage{wrapfig}
\usepackage{rotating}
\usepackage[normalem]{ulem}
\usepackage{amsmath}
\usepackage{textcomp}
\usepackage{amssymb}
\usepackage{capt-of}
\usepackage{hyperref}
\author{Gleefre}
\date{\today}
\title{Package-Local Nicknames}
\hypersetup{
 pdfauthor={Gleefre},
 pdftitle={Package-Local Nicknames},
 pdfkeywords={},
 pdfsubject={This is a CDR specification for package-local nicknames.},
 pdfcreator={Emacs 27.1 (Org mode 9.3)}, 
 pdflang={English}}
\begin{document}

\maketitle
\tableofcontents

[THIS IS A DRAFT]

\section{ISSUES}
\label{sec:orgfe5250b}
\subsection{Issue 1.}
\label{sec:org24771a2}
Inconsistency between return values of \texttt{add-package-local-nickname} and
\texttt{remove-package-local-nickname}. First always returns \emph{designated package},
second one returns \emph{true} if a nickname was removed (but it is not specified
what exactly is returned). (Somewhat similar functions from the standard -
\texttt{use-package} and \texttt{unuse-package} always return \texttt{T} - their api is
consistent.)
\subsubsection{Proposal}
\label{sec:orgc72ed94}
\begin{itemize}
\item \texttt{add-package-local-nickname} should return \emph{designated package} if a new
nickname was added and \texttt{NIL} otherwise (if it already existed).
\item \texttt{remove-package-local-nickname} should return \emph{designated package} if
nickname was removed and \texttt{NIL} otherwise.
\end{itemize}
\subsubsection{Currently}
\label{sec:org7a605f4}
sbcl, ccl, ecl, acl, abcl, clasp: \texttt{add-package-local-nickname} always returns
\emph{designated package}.

sbcl, ccl, ecl, acl, clasp: \texttt{remove-package-local-nickname} returns \texttt{T} on
success and \texttt{NIL}

abcl: \texttt{remove-package-local-nickname} returns package nicknamed by the
removed nickname on success and \texttt{NIL}.  [ That seems nice ]
\subsection{Issue 2 (PRINT-READ consistency)}
\label{sec:orga89533d}
Lisp reader uses \texttt{find-package} to read a symbol, and is affected by \emph{local
nicknames} of the \emph{current package}. So in order to maintain \textbf{print-read}
consistency it is required to use a correct \emph{package prefix} - such prefix
that calling \texttt{find-package} on it in the \emph{current package} will return the
symbol's \emph{home package}.

There are several situations to consider:
\begin{enumerate}
\item There \textbf{is} a \emph{local nickname} defined in the \emph{current package} for the
symbol's \emph{home package}.

\emph{In this case such local nickname can be used as the package prefix.}
\item Symbol's home \emph{package name} or one of its \emph{global nicknames} is not
shadowed by any \emph{local nickname} defined in the \emph{current package}.

\emph{In this case that package name or global nickname can be used as the
package prefix.}
\item Symbol's home \emph{package name} and all its \emph{global nicknames} are shadowed by
one of the \emph{local nicknames} of the \emph{current package} and there \textbf{is no}
\emph{local nickname} defined (in the \emph{current package}) for the symbol's home
package.

\emph{It is not clear what should be done in this case (see proposals).}
\end{enumerate}
\subsubsection{Example}
\label{sec:orge8889aa}
\begin{verbatim}
(defpackage #:a (:use) (:export #:+))
(defpackage #:b (:local-nicknames (#:a #:cl)))
(let ((*package* (find-package '#:b)))
  (print 'a:+))
; => a:+ everywhere
;; but in #:b package a:+ would refer to cl:+
\end{verbatim}
\begin{verbatim}
(defpackage #:a (:use) (:export #:+))
(defpackage #:b (:use) (:export #:+))
(defpackage #:c (:use) (:local-nicknames (#:a #:b) (#:b #:a)))
(let ((*package* (find-package '#:c)))
  (print 'a:+))
; => b:+  (sbcl, ccl, abcl)
; => a:+  (clasp, acl, ecl)
;; but in #:c package a:+ would refer to b:+
\end{verbatim}
\subsubsection{Currently}
\label{sec:org9a3dd4e}
sbcl, ccl, abcl: print symbol with package name when all package's names
and nicknames are shadowed by \emph{current package}'s \emph{local nicknames}.

ecl, acl, clasp: don't print with local nicknames at all.
\subsubsection{Proposals}
\label{sec:org82b6e84}
\begin{itemize}
\item The symbol must be printed using the \texttt{\#.} syntax:
\begin{verbatim}
#.(cl:let ((cl:*package* (cl:find-package "KEYWORD")))
    (cl:find-symbol "BAR" "FOO"))
;; or
#.(cl:let ((cl:*package* (cl:find-package "KEYWORD")))
    (cl:intern "BAR" "FOO"))
\end{verbatim}
Note that \texttt{\#:KEYWORD} name is reserved for the \texttt{\#:KEYWORD} package and
cannot be used as a \emph{local nickname} thus this expression will always
evaluate to the symbol \texttt{foo::bar}.
\item \emph{Shinmera's idea}. In this case an extended \texttt{\#:} syntax should be used:
\begin{verbatim}
#:(package name) and #::(package name)
\end{verbatim}
\item In this case the symbol must be printed using the \texttt{\#`} syntax for reading
an expression ignoring \emph{local nicknames} in the \emph{current package}:
\begin{verbatim}
#`foo:bar  and  #`foo::bar
\end{verbatim}


It can be implemented roughly as follows:
\begin{verbatim}
(defun |#`-reader| (stream subchar arg)
  (declare (ignore subchar arg))
  (let* ((current-package *package*)
         (local-nicknames (package-local-nicknames current-package)))
    (loop for (nick . package) in local-nicknames
          do (remove-package-local-nickname nick current-package))
    (unwind-protect
      (read stream t nil t)
      (loop for (nick . package) in local-nicknames
            do (add-package-local-nickname nick package current-package)))))

(set-dispatch-macro-character #\# #\` #'|#`-reader|)
\end{verbatim}
It is implementation dependent whether \emph{local nicknames} are actually
removed from the \emph{current package} or not.
\item In this case the symbol must be printed unreadably (specifics are
implementation dependent):
\begin{verbatim}
#<SYMBOL IN THE SHADOWED PACKAGE FOO:BAR>
#<SYMBOL IN THE SHADOWED PACKAGE FOO::BAR>
\end{verbatim}

If \texttt{*print-readably*} is \emph{true} must signal an error of type
\texttt{print-not-readable} without printing anything.
\item In this case the symbol must be printed using \texttt{:::} and \texttt{::::} syntax
to lookup and intern ignoring \emph{local nicknames} respectively:
\begin{verbatim}
foo:::bar  ; same as (cl:find-symbol "BAR" "FOO") in the #:KEYWORD package
foo::::bar  ; same as (cl:intern "BAR" "FOO") in #:KEYWORD package
\end{verbatim}
\end{itemize}
\subsection{Issue 3}
\label{sec:org899499b}
It is not clear whether \emph{local nicknames} of the \emph{current package} should
affect \texttt{make-package} or \texttt{defpackage}.
\subsubsection{Example}
\label{sec:org860871b}
\begin{verbatim}
(defpackage #:a (:use) (:export #:x))
(defpackage #:b (:use) (:export #:x))
(defpackage #:c
  (:use #:cl)
  (:local-nicknames (#:a #:b) (#:b #:a)))

(in-package #:c)
(defpackage #:d
  (:use #:a)
  (:export #:x))
(print (package-name (symbol-package 'd:x)))
; => "B"  (sbcl, ccl, acl, abcl)
; => "A"  (ecl, clasp)

(defpackage #:e
  (:use)
  (:local-nicknames (#:x #:a)))

(let ((*package* (find-package '#:e)))
  (print (package-name (find-package '#:x))))
; => "B"  (ccl, acl, abcl)
; => "A" (sbcl, ecl, clasp)
\end{verbatim}
\subsubsection{Proposal}
\label{sec:orgcf7e4a2}
They should affect all options like \texttt{:use}, \texttt{:local-nicknames},
\texttt{:shadowing-import-from} and \texttt{:import-from}.
\subsubsection{Currently}
\label{sec:org489a733}
ccl, acl, abcl: they do.

sbcl: partly (:use is affected, :local-nicknames are not)

ecl, clasp: they don't.
\subsection{Issue 4}
\label{sec:orgdc86dd9}
It is not clear whether \emph{local nicknames} of the package \textbf{being defined}
should affect \texttt{make-package} or \texttt{defpackage}.
\subsubsection{Example}
\label{sec:orgb30b0f7}
\begin{verbatim}
(defpackage #:a (:use) (:export #:x))
(defpackage #:b (:use) (:export #:x))
(defpackage #:c
  (:use #:a)
  (:local-nicknames (#:a #:b) (#:b #:a)))

(print (package-name (symbol-package 'c::x)))
; => "A"  (sbcl, ccl, acl, abcl, clasp)
;; ecl errors... :/
\end{verbatim}
\subsubsection{Proposal}
\label{sec:org821a03c}
They should not.
\subsubsection{Currently}
\label{sec:org18f74c8}
sbcl, ccl, acl, abcl: they don't.

ecl: they do (or not? not sure).

clasp: they do in some weird way - also depends on order in which PLN are
added.
\subsection{Issue 5}
\label{sec:org82cdd43}
It is not clear whether it is valid to have a \emph{local nickname} in a package
shadowing its own name or nickname.

Now: SBCL signals an error, but that doesn't make much sense, especially if a
\emph{local nickname} shadows the \emph{global nickname} of a package and not its
name. And it is possible to achieve a \emph{local nickname} shadowing a \emph{global
nickname} or the \emph{package name} by renaming the package.
\subsubsection{Example}
\label{sec:org689b58e}
\begin{verbatim}
(defpackage #:a (:use) (:local-nicknames (#:a #:cl)))
; => error  (sbcl, ccl, abcl)
; => ok  (ecl, acl, clasp)
\end{verbatim}
\subsubsection{Proposal}
\label{sec:orgc6ee91f}
It should be allowed, but a warning might be signaled.
\subsubsection{Currently}
\label{sec:org3d9a91f}
sbcl, ccl, abcl: not allowed.

ecl, acl, clasp: allowed.
\subsection{Proposal 6}
\label{sec:orga337064}
Add \texttt{:local-nicknames} option to \texttt{make-package} similar to \texttt{defpackage}. See
\hyperref[sec:orgba122fd]{make-package}.
\subsubsection{Currently}
\label{sec:orgf1b9355}
ecl: takes keyword parameter, but segfaults on incorrect usage + doesn't
have it in docstring. Also API differs :/

acl: has it

abcl, sbcl, ccl, clasp: don't have it.
\subsection{Issue 7}
\label{sec:org88112c4}
It is not clear whether functions \texttt{package-local-nicknames} and
\texttt{package-locally-nicknamed-by-list} should be allowed to return lists with
duplicate entries.
\subsubsection{Example}
\label{sec:orgefc34eb}
\begin{verbatim}
(defpackage #:a (:use) (:local-nicknames (#:b #:cl) (#:c #:cl)))
(print (mapcar #'package-name (package-locally-nicknamed-by-list '#:cl)))
; => ("A")  (sbcl, acl, clasp)
; => ("A" "A")  (ccl, ecl, abcl)
\end{verbatim}
\subsubsection{Proposal}
\label{sec:orgbbb2690}
They should not.
\subsubsection{Currently}
\label{sec:org342899c}
sbcl, acl, clasp, ccl, abcl, ecl: don't return duplicate entries from \texttt{package-local-nicknames}.

sbcl, acl, clasp: don't return duplicate entries from \texttt{package-locally-nicknamed-by-list}.

ccl, abcl, ecl: return duplicate entries from \texttt{package-locally-nicknamed-by-list}.
\subsection{Issue 8 \emph{by |3b|}}
\label{sec:org3661e59}
How PLN affect \texttt{format}'s \texttt{\textbackslash{}\textasciitilde{}//} directive? It seems tricky - compilers might
want optimize it at compile time, but reffered symbol might change on
rebinding the \texttt{*package*}.
\subsubsection{Example}
\label{sec:orgf6f5e81}
\begin{verbatim}
;;;; File test.lisp
(defpackage #:a (:use) (:export #:ff))
(defpackage #:b (:use) (:export #:ff))

(defun a:ff (stream &rest args)
  (declare (ignore args))
  (format stream "A:FF"))

(defun b:ff (stream &rest args)
  (declare (ignore args))
  (format stream "B:FF"))

(defpackage #:foo
  (:use #:cl)
  (:local-nicknames (#:nick #:a)))

(defpackage #:bar
  (:use #:cl)
  (:local-nicknames (#:nick #:b)))

(in-package #:foo)
(defun test ()
  (format t "Called ~/nick:ff/ & " nil)
  (let ((*package* (find-package '#:foo)))
    (format t "~/nick:ff/~%" nil)))

(test)
; => "Called A:FF & A:FF"  (sbcl, ccl, acl, abcl, ecl, clasp)

(let ((*package* (find-package '#:bar)))
  (test))
; => "Called A:FF & A:FF"  (sbcl, clasp)
; => "Called B:FF & A:FF"  (ccl, acl, abcl, ecl)
\end{verbatim}
\section{Introduction}
\label{sec:orgd929a2e}
This is a specification for package-local nicknames extension in Common Lisp.
\subsection{Rationale}
\label{sec:org2266ce4}
Package-local nicknames allow to use short and easy-to-use names without
potentially introducing name conflict as with normal nicknames.
\subsection{Current state}
\label{sec:org3ba054c}
Package-local nicknames are implemented (at least partially) in \texttt{SBCL},
\texttt{CCL}, \texttt{ECL}, \texttt{Clasp}, \texttt{ABCL}, \texttt{Allegro CL}, \texttt{LispWorks}. Unfortunately,
there are multiple inconsistencies between implementations, and all
implementations lose \textbf{print-read} consistency to some extent.
\subsection{Goal}
\label{sec:org6e1c355}
The purpose of this document is to standardize the package-local nicknames
extension and to address some existing issues (mostly \textbf{print-read}
consistency).

[TODO]

This CDR also aims to provide an extensive test suite for this extension.
\section{Specification}
\label{sec:orgde33d6a}
\subsection{Description}
\label{sec:org8b65d64}
\subsubsection{Concept}
\label{sec:orgacc645b}
\emph{Package-local nickname} (or \emph{local nickname}) has the same effects as a
normal \emph{package nickname} (later \emph{global nickname}), except that these
effects only apply when \texttt{*package*} is bound to a package for which the
nickname has been defined.

That means that calls to \texttt{find-package} with a \emph{local nickname} defined in
the \emph{current package} should return the package nicknamed by this nickname.

This also affects all implied calls to \texttt{find-package}, including those
performed by the lisp reader.

In addition, to maintain \textbf{print-read} consistency, the lisp printer is
affected by \emph{local nicknames} defined in the \emph{current package}, for details
see \hyperref[sec:orga89533d]{PRINT-READ consistency}.

\emph{Local nickname} is allowed to shadow a \emph{package name} or a \emph{global
nickname}, except for the names \texttt{\#:CL}, \texttt{\#:COMMON-LISP} and \texttt{\#:KEYWORD}
which should always refer to their packages.
\subsubsection{API}
\label{sec:org3d90c8b}
\noindent\rule{\textwidth}{0.5pt}
\begin{enumerate}
\item defpackage
\label{sec:org2125320}
\texttt{defpackage} options are extended to include \emph{local-nicknames-option}:
\begin{verbatim}
local-nicknames-option ::= (:local-nicknames (nickname package)*)
\end{verbatim}


Each pair specifies a \emph{local nickname} \texttt{nickname} for the corresponding
\texttt{package}.

This option may appear more than once.
\begin{enumerate}
\item Arguments and Values:
\label{sec:org01f8a40}
\texttt{nickname} must be a \emph{string designator}.

\texttt{package} must be a \emph{package designator}.
\item Exceptional situations
\label{sec:org43b8ddd}
An error of type \texttt{package-error} is signaled when a package designated by
\texttt{package} does not exists.

Name conflict errors are handled by the underlying calls to
\texttt{add-package-local-nickname}.

See \hyperref[sec:org25cb4a3]{add-package-local-nickname: exceptional situations}.
\item Implementation dependent
\label{sec:org8c9a77d}
The behaviour is unspecified when a \emph{local nickname} is specified for the
package that is being defined.

The behaviour is unspecified when supplied \emph{local nicknames} are at
variance with the current state of the package that is being defined. An
implementation might choose to remove all present \emph{local nicknames} at the
begining of each redefinition of the package.

[TODO: What happens when a package is redefined with local
nicknames in other packages that it is nicknamed by? It probably
can't be strictly defined since redefining package is
implementation dependent\ldots{} But seems like they must be left
intact.]

\noindent\rule{\textwidth}{0.5pt}
\end{enumerate}
\item make-package
\label{sec:orgba122fd}
\hyperref[sec:orga337064]{Proposal\#6}

\texttt{make-package} lambda list is extended to include an additional key
parameter: \texttt{local-nicknames}.
\begin{verbatim}
local-nicknames ::= ((nickname package)*)
\end{verbatim}


\texttt{local-nicknames} defaults to an \emph{empty list}.

\texttt{local-nicknames} must be a \emph{list} each element of which must be a \emph{list}
of form \texttt{(nickname package)}. Specifies \emph{local nicknames} in the new
\emph{package}.
\begin{enumerate}
\item Arguments and Values:
\label{sec:orge3b1768}
\texttt{local-nicknames} must be a a \emph{list} of pairs \texttt{(nickname package)}.

\texttt{nickname} must be a \emph{string designator}.

\texttt{package} must be a \emph{package designator}.
\item Exceptional situations
\label{sec:org6dc9df4}
An error of type \texttt{package-error} is signaled when a package designated by
\texttt{package} does not exists.

Name conflict errors are handled by the underlying calls to
\texttt{add-package-local-nickname}.

See \hyperref[sec:org25cb4a3]{add-package-local-nickname: exceptional situations}.
\item Implementation dependent
\label{sec:org80ab67e}
The behaviour is unspecified when a \emph{local nickname} is specified for the
package that is being defined.

\noindent\rule{\textwidth}{0.5pt}
\end{enumerate}
\item add-package-local-nickname
\label{sec:org5514b8b}
\begin{verbatim}
(add-package-local-nickname nickname actual-package &optional designated-package)
  => designated-package-object
\end{verbatim}

\texttt{designated-package} defaults to the \emph{current package}.

Adds a \emph{package-local nickname} \texttt{nickname} for the \texttt{actual-package} in the
\texttt{designated-package}.

Returns the package designated by \texttt{designated-package}.

If a \emph{nickname} is already defined, checks that it is defined for the
package designated by \texttt{actual-package}.
\begin{enumerate}
\item Arguments and Values
\label{sec:org40ae2bf}
\texttt{nickname} must be a \emph{string designator}.

\texttt{actual-package} and \texttt{designated-package} must be \emph{package designators}.

\texttt{designated-package-object} is of type \emph{package}.
\item Exceptional situations
\label{sec:org25cb4a3}
If a package designated by \texttt{actual-package} or a package designated by
\texttt{designated-package} does not exists, an error of type \emph{package-error}
must be signaled.

If \texttt{nickname} is one of the names \texttt{\#:CL}, \texttt{\#:COMMON-LISP} or \texttt{\#:KEYWORD},
an error of type \emph{package-error} must be signaled.

If \texttt{nickname} is a \emph{local nickname} for a package different from
\texttt{actual-package}, an error of type \emph{package-error} must be signaled.
\item Implementation dependent
\label{sec:orgdfabbde}
\textbf{PROPOSAL} (See \hyperref[sec:orgfe5250b]{issues\#4}.)

If \texttt{nickname} shadows the \texttt{designated-package}'s \emph{package name} or one of
its \emph{global nicknames}, a style warning might signaled.

\noindent\rule{\textwidth}{0.5pt}
\end{enumerate}
\item remove-package-local-nickname
\label{sec:orgbb9670c}
\begin{verbatim}
(remove-package-local-nickname old-nickname &optional designated-package)
  => nickname-removed-p
\end{verbatim}

\texttt{designated-package} defaults to the \emph{current package}.

If \texttt{designated-package} has \texttt{old-nickname} as a \emph{local nickname}, it is
removed.

Returns \emph{true} if the \texttt{old-nickname} existed (and was removed), and \texttt{NIL}
otherwise.
\begin{enumerate}
\item Arguments and Values
\label{sec:org59b00ee}
\texttt{old-nickname} must be a \emph{string designator}.

\texttt{designated-package} must be a \emph{package designator}.

\texttt{nickname-removed-p} is a \emph{generalized boolean}.
\item Exceptional situations
\label{sec:org23f5e76}
If a package designated by \texttt{designated-package} does not exists, an error of
type \emph{package-error} must be signaled.

\noindent\rule{\textwidth}{0.5pt}
\end{enumerate}
\item package-local-nicknames
\label{sec:org76aac7b}
\begin{verbatim}
(package-local-nicknames package)
  => local-nicknames-alist
\end{verbatim}

Returns an \emph{alist} describing local nicknames defined in a package
designated by \texttt{package}.

Each cons cell in \texttt{local-nicknames-alist} is of the form \texttt{(nickname . package)}
where \texttt{nickname} is of type \emph{string} and \texttt{package} is of type
\emph{package}.
\begin{enumerate}
\item Arguments and Values
\label{sec:org7e527fb}
\texttt{package} must be a \emph{package designator}.

\texttt{local-nicknames-alist} is an \emph{alist} with keys of type \emph{string} and
values of type \emph{package}.
\item Exceptional situations
\label{sec:orgda6091c}
An error of type \texttt{package-error} is signaled when a package designated by
\texttt{package} does not exists.
\item Notes
\label{sec:org1770de4}
The returned \emph{alist} must be safe to be modified by the user.

\noindent\rule{\textwidth}{0.5pt}
\end{enumerate}
\item package-locally-nicknamed-by-list
\label{sec:orgf7c6b82}
\begin{verbatim}
(package-locally-nicknamed-by-list package)
  => packages-list
\end{verbatim}

Returns a \emph{list} of packages that have a \emph{local nickname} defined for the
package designated by \texttt{package}.
\begin{enumerate}
\item Arguments and Values
\label{sec:orga92d1c6}
\texttt{package} must be a \emph{package designator}.

\texttt{packages-list} is a \emph{list} with elements of type \emph{package}.
\item Exceptional situations
\label{sec:orgece0c8a}
An error of type \texttt{package-error} is signaled when a package designated by
\texttt{package} does not exists.
\item Notes
\label{sec:org9f4c5d3}
The returned \emph{list} must be safe to be modified by the user.

\noindent\rule{\textwidth}{0.5pt}
\end{enumerate}
\end{enumerate}
\subsubsection{Affected symbols}
\label{sec:orgfccbc2c}
\noindent\rule{\textwidth}{0.5pt}
\begin{enumerate}
\item defpackage
\label{sec:org883e09b}
See \hyperref[sec:org2125320]{defpackage}.

\noindent\rule{\textwidth}{0.5pt}
\item make-package
\label{sec:org9a9c76d}
See \hyperref[sec:orgba122fd]{make-package}.

\noindent\rule{\textwidth}{0.5pt}
\item find-package
\label{sec:org3b5a5ff}
When argument to \texttt{find-package} is a \emph{local nickname} that is defined in
the \emph{current package}, returns the package corresponding to this nickname.

This also affects all implied calls to \texttt{find-package}, including but not
limited to those performed by the lisp reader as well as those performed by
\texttt{export}, \texttt{find-symbol}, \texttt{import}, \texttt{rename-package}, \texttt{shadow},
\texttt{shadowing-import}, \texttt{delete-package}, \texttt{with-package-iterator}, \texttt{unexport},
\texttt{unintern}, \texttt{in-package}, \texttt{unuse-package}, \texttt{use-package}, \texttt{do-symbols},
\texttt{do-external-symbols}, \texttt{do-all-symbols}, \texttt{intern}, \texttt{package-name},
\texttt{package-nicknames}, \texttt{package-shadowing-symbols}, \texttt{package-use-list},
\texttt{package-used-by-list}.

\texttt{add-package-local-nickname}, \texttt{remove-package-local-nickname},
\texttt{package-local-nicknames} and \texttt{package-locally-nicknamed-by} are also
affected.

There are two exceptions: \texttt{make-package} and \texttt{defpackage} must \textbf{not} be
affected by \emph{local nicknames} of the \emph{current package}.

\noindent\rule{\textwidth}{0.5pt}
\item rename-package
\label{sec:org6bcd57c}
When a package is renamed via \texttt{rename-package} it maintains all \emph{local
nicknames} it is nicknamed by, as well as all \emph{local nicknames} it has
defined.
\begin{enumerate}
\item Implementation dependent
\label{sec:org1b49a57}
\textbf{PROPOSAL} (See \hyperref[sec:orgfe5250b]{issues\#4}.)

If a \emph{new-name} or one of \emph{new-nicknames} is shadowed by one of the \emph{local
nicknames} of the package being redefined, a warning might be signaled.

\noindent\rule{\textwidth}{0.5pt}
\end{enumerate}
\item delete-package
\label{sec:org4dc3c88}
When a package is deleted via \texttt{delete-package} all \emph{local nicknames}
defined in other packages that it was nicknamed by must be removed as well
as all \emph{local nicknames} defined in the package that is being deleted.

This also means that this package must not be available by calls to
\texttt{package-locally-nicknamed-by-list} and \texttt{package-local-nicknames}.

\noindent\rule{\textwidth}{0.5pt}
\end{enumerate}
\subsubsection{\texttt{*FEATURES*}}
\label{sec:org668cb7e}
If an implementation supports package-local nicknames it should add symbols
\texttt{:package-local-nicknames} and \texttt{:cdr-15} (per CDR 14) to \texttt{*features*}.
\subsection{Examples}
\label{sec:org0c457a2}
[TODO]
\section{Links}
\label{sec:orge0c97f1}
3b's \href{https://github.com/3b/package-local-nicknames/blob/master/docs.org}{notes} on package-local nicknames.

phoe's \href{https://github.com/phoe/trivial-package-local-nicknames}{tests}.

SBCL's \href{https://www.sbcl.org/manual/\#Package\_002dLocal-Nicknames}{manual entry}.
\section{Copying and License}
\label{sec:org06baf55}
[TODO]
\end{document}
